\documentclass[a4paper,12pt]{article}

%%% Работа с русским языком

\usepackage{cmap}					% поиск в PDF
\usepackage{mathtext} 				% русские буквы в формулах
\usepackage[T2A]{fontenc}			% кодировка
\usepackage[utf8]{inputenc}			% кодировка исходного текста
\usepackage[english,russian]{babel}	% локализация и переносы
\usepackage{indentfirst}            % красная строка в первом абзаце
\usepackage[unicode]{hyperref}
\usepackage{epigraph}
\frenchspacing                      % равные пробелы между словами и предложениями

%%% Дополнительная работа с математикой
\usepackage{amsmath,amsfonts,amssymb,amsthm,mathtools} % пакеты AMS
\usepackage{bbm} % Blackboard bold для цифр
\usepackage{icomma}                                    % "Умная" запятая

\renewcommand{\phi}{\ensuremath{\varphi}}
\renewcommand{\kappa}{\ensuremath{\varkappa}}
\renewcommand{\le}{\ensuremath{\leqslant}}
\renewcommand{\leq}{\ensuremath{\leqslant}}
\renewcommand{\ge}{\ensuremath{\geqslant}}
\renewcommand{\geq}{\ensuremath{\geqslant}}
\renewcommand{\emptyset}{\ensuremath{\varnothing}}

\newcommand{\cl}{\text{cl }}
\newcommand{\setint}{\text{int }}

\theoremstyle{plain}
\newtheorem{theorem}{Теорема}[section]
\newtheorem{lemma}{Лемма}[section]
\newtheorem{proposition}{Утверждение}[section]
\newtheorem*{corollary}{Следствие}
\newtheorem*{exercise}{Упражнение}

\theoremstyle{definition}
\newtheorem{definition}{Определение}[section]
\newtheorem*{note}{Замечание}
\newtheorem*{reminder}{Напоминание}
\newtheorem*{example}{Пример}
\newtheorem*{tasks}{Вопросы и задачи}

\theoremstyle{remark}
\newtheorem*{solution}{Решение}

%%% Оформление страницы
\usepackage{extsizes}     % Возможность сделать 14-й шрифт
\usepackage{geometry}     % Простой способ задавать поля
\usepackage{setspace}     % Интерлиньяж
\usepackage{enumitem}     % Настройка окружений itemize и enumerate
\setlist{leftmargin=25pt} % Отступы в itemize и enumerate

\geometry{top=25mm}    % Поля сверху страницы
\geometry{bottom=30mm} % Поля снизу страницы
\geometry{left=20mm}   % Поля слева страницы
\geometry{right=20mm}  % Поля справа страницы

\begin{document}
\tableofcontents
\newpage

\section{Основные понятия, простейшие типы дифференциальных уравнений}
\subsection{Основные понятия}
\begin{definition}
	Уравнение вида
	\begin{equation}
		\label{SDE}
		F(x,\, y,\, \ldots,\, y^{(n)}) = 0
	\end{equation}

	называется обыкновенным дифференциальным уравнением $n$-го порядка.
\end{definition}

\begin{definition}
	Функция $\phi(x)$, определённая на $I$ вместе со своими $n$ производными, называется решением уравнения (\ref*{SDE}), если:

	\begin{enumerate}
		\item $\phi$ и все её $n$ производных непрерывны на $I$.
		\item $\forall x \in I:\: (x,\, \phi(x),\, \phi'(x),\, \ldots,\, \phi^{(n)}(x)) \in \Omega$, где $\Omega$ - область определения $F$.
		\item $\forall x \in I:\: F(x,\, \phi(x),\, \phi'(x),\,\ldots,\,\phi^{(n)}) = 0$
	\end{enumerate}
\end{definition}

\begin{definition}
	Решение $y = \phi(x),\, x \in \langle a,\, b\rangle$ уравнения (\ref*{SDE}) называется продолжаемым вправо, если существует такое решение $y = \psi(x),\, x \in \langle a,\, b_1 \rangle,\, \langle a,\, b\rangle \subset \langle a,\, b_1 \rangle$, что $\phi(x) \equiv \psi(x)$ при $x \in \langle a,\, b\rangle$

	Аналогично определяется продолжение решения влево.
\end{definition}

\begin{definition}
	Решение называется непродолжаемым, если его нельзя продолжить ни вправо, ни влево.
\end{definition}

\begin{definition}
	Система дифференциальных уравнений называется автономной, если она имеет вид:
	\[
		\frac{dx_k}{dt} = f_k(x_1,\, \ldots,\,x_n);\; k = 1,\,\ldots,\,n
	\]
	также очень часто автономные системы записываются в компактном векторном виде:
	\begin{equation}
		\label{Autonomius}
		\dot{x} = F(x),\, x \in \Omega \subseteq E^n
	\end{equation}

\end{definition}

\begin{definition}
	Непрерывно дифференцируемая в $\Omega$ функция $u(x)$ называется первым интегралом системы $(\ref*{Autonomius})$, если $\forall t \in T:\: u(x(t)) \equiv const$ для каждого решения $x(t)$ этой системы.
\end{definition}

\begin{definition}
	Если поставить в соответствие каждой точке $(x,\, y)$ некоторого множества $\Omega \subseteq E^2$ вектор с координатными представлением $(1,\, f(x,\,y))$, то полученное векторное множество принято называть полем направлений ОДУ первого порядка.
\end{definition}

\begin{definition}
	Векторное поле - это отображение, которое сопоставляет каждой точке некоторого пространства вектор
\end{definition}

\begin{definition}
	Пусть $x(t)$ есть частное решение системы (\ref*{Autonomius}), тогда вектор-функция $x(t),\, t \in T$, параметрически задаёт некоторую линию в $E^n$, называемую фазовой траекторией этой системы.
\end{definition}

\begin{definition}
	Совокупность фазовых траекторий для всех частных решений будем именовать фазовым портретом системы (\ref*{Autonomius})
\end{definition}

\begin{definition}
	График функции $y = \phi(x)$ можно рассматривать как геометрическое представление частного решения уравнения (\ref*{SDE}). Этот график обычно называют интегральной кривой уравнения (\ref*{SDE}).
\end{definition}

\subsection{Простейшие типы уравнений первого порядка}

\subsubsection*{Уравнения с разделяющимися переменными}
\begin{definition}
	Уравнения с разделяющимися переменными - это уравнения, которые могут быть записаны в виде

	\[y' = f(x)g(y)\; f(x) \in C(I_1),\, g(y) \in C(I_2)\]

	или же в виде

	\[M(x)N(y)dx + P(x)Q(y)dy = 0\]
\end{definition}

\begin{note}
	Если же $y_k \in I_2$ решение уравнения $g(y) = 0$, то $y \equiv y_k$ -- решение дифф. уравнения
\end{note}

Если же $y(x)$ нигде не принимает значение $y_k$, то $g(y) \neq 0$, а потому мы можем делить на него. Значит, чтобы решить исходное уравнение, необходимо разделить переменные, то есть, привести уравнение к такой форме, чтобы при дифференциале $dx$ стояла функция, зависящая лишь от $x$, а при дифференциале $dy$ -- функция, зависящая от $y$.

\subsubsection*{Однородные уравнения}
\begin{definition}
	Функция двух переменных $f(x,\,y)$ называется однородной степени $m$, если для всех $t$ справедливо соотношение:
	\[f(tx,\,ty) = t^mf(x,\,y)\]
\end{definition}

\begin{definition}
	Однородным дифференциальным уравнением называется уравнение вида
	\[M(x,\,y)dx + N(x,\,y)dy = 0\]
	если $M(x,\,y)$ и $N(x,\,y)$ -- однородные функции одной и той же степени $m$.
\end{definition}

Однородное уравнение приводится к уравнению с разделяющимися переменными с помощью замены искомой функции $y(x)$ по формуле:
\[t(x) = \frac{y(x)}{x}\]

Тогда производная $y'$ и дифференциал $dy$ заменяются по формулам:
\[y' = t'x + t,\,\, dy = tdx + xdt\]

\subsubsection*{Линейные уравнения}
\begin{definition}
	Линейным уравнением первого порядка называется уравнение, линейное относительно искомой функции $y(x)$ и её производной, то есть, уравнения вида
	\[y' + a(x)y = b(x)\;\;\; a(x),\,b(x) \in C(I)\]

	Функция $b(x)$ называется свободным членом уравнения.

	Уравнение \[y' + a(x)y = 0\] называется линейным однородным уравнением, соответствующим изначальному линейному уравнению.
\end{definition}

Покажем, что однородное уравнение является уравнением с разделяющимися перменными

\[y' + a(x)y = 0 \Rightarrow \int \frac{dy}{y} = -\int a(x) dx \Rightarrow |y| = e^C \cdot e^{-\int_{x_0}^x a(t) dt}\]

Объединяя все решения, получаем общее решение:
\[y_0 = C \exp\left[-\int_{x_0}^x a(t) dt\right]\]

Будем искать частное решение исходного линейного уравнения методом вариации постоянной:
\[y_{\textbf{ч}} = C(x) \cdot \exp\left[-\int_{x_0}^x a(t) dt\right]\]

\subsubsection*{Уравнение Бернулли}
\begin{definition}
	Нелинейное уравнение первого порядка вида
	\[y' + a(x)y = b(x)y^m,\;\;\; m \neq 0,\, m \neq 1,\, a,\,b \in C(I)\]
	называется уравнением Бернулли.
\end{definition}

Заметим, что $y = 0$ -- решение уравнения Бернулли при $m > 0$.

Если $y \neq 0$, то, разделив уравнение на $y^m$ и вводя новую неизвестную функцию $z = y^{1 - m}$, относительно функции $z$ получаем линейное уравнение.

\subsubsection*{Уравнение Рикатти}
\begin{definition}
	Нелинейное уравнение первого порядка вида
	\[y' = a(x)y^2 + b(x)y + c(x) \;\;\; a,\,b\,\,c \in C(I)\]
	называется уравнением Рикатти
\end{definition}

В отличие от всех уравнений, рассматривавшихся ранее, уравнение Рикатти не всегда интегрируется в квадратурах. Чтобы решить его, необходимо знать хотя бы одно частное решение $y = y_1(x)$ этого уравнения. Тогда замена $y = y_1 + z$ приводит это уравнение к уравнению Бернулли.

\subsubsection*{Логистическое уравнение Ферхюльста}
\begin{note}
	Исходные предположения для вывода уравнения при рассмотрении популяционной динамики выглядит следующим образом:
	\begin{itemize}
		\item Скорость размножения популяции пропорциональна её текущей численности при прочих равных условиях
		\item Скорость размножения популяции пропорциональна количеству доступных ресурсов при прочих равных условиях.
	\end{itemize}
\end{note}

\begin{definition}
	Обозначая через $P$ численность популяции, а время - $t$, модель можно свести к дифференциальному уравнению
	\[\frac{dP}{dt} = rP(1 - \frac{P}{K})\]
	где параметр $r$ характеризует скорость роста, а $K$ -- максимальную возможную численность популяции.
\end{definition}

\begin{note}
	Точным решения является логистическая функция, S-образная кривая:
	\[P(t) = \frac{KP_0e^{rt}}{K + P_0(e^{rt}-1)}\] где $P_0$ -- начальная популяция, и $\lim\limits_{t \to \infty} P(t) = K$.
\end{note}

\subsection{Уравнения в полных дифференциалах, интегрирующий множитель}
\subsubsection*{Уравнения в полных дифференциалах}
\begin{definition}
	Это уравнение
	\[M(x,\,y)dx + N(x,\,y)dy = 0\]
	называется уравнением в полных дифференциалах, если его левая часть является дифференциалом некоторой гладкой функции $F(x,\,y)$. Тогда это уравнение можно переписать в виде $dF(x,\,y) = 0$, так что его решение будет иметь вид
	\[F(x,\,y) = C\]
\end{definition}

\begin{proposition}
	Если функции $M(x,\,y)$ и $N(x,\,y)$ определены и непрерывны в некоторой односвязной области $\Omega$ и имеют в ней непрерывные частные производные по $x$ и по $y$, то изначальное уравнение будет уравнением в полных дифференциалах тогда и только тогда, когда выполняется тождество
	\[\frac{\partial M(x,\,y)}{\partial y} \equiv \frac{\partial N(x,\,y)}{\partial x}\]
\end{proposition}

\subsubsection*{Интегрирующий множитель}
Пусть дано уравнение в дифференциалах, которое не является уравнением в полных дифференциалах.

\begin{definition}
	Функция $\mu(x,\,y) \neq 0$ называется интегрирующим множителем для исходного уравнения, если уравнение
	\[\mu(x,\,y)(M(x,\,y)dx + N(x,\,y)dy) = 0\]
	является уравнением в полных дифференциалах. Отсюда следует, что функция $\mu$ удовлетворяет условию
	\[\frac{\partial(\mu M)}{\partial y} \equiv \frac{\partial(\mu N)}{\partial x}\]
	Это равенство даёт уравнение в частных производных первого порядка для $\mu(x,\,y):$
	\[N\frac{\partial\mu}{\partial x} - M\frac{\partial\mu}{\partial y} = \left(\frac{\partial M}{\partial y} - \frac{\partial N}{\partial x}\right)\mu\]
	Поделив обе части последнего уравнения на $\mu$, перепишем его в виде:
	\[N\frac{\partial \ln\mu}{\partial x} - M\frac{\partial \ln\mu}{\partial y} = \frac{\partial M}{\partial y} - \frac{\partial N}{\partial x}\]
\end{definition}

\subsubsection*{Точные и замкнутые 1-формы, лемма Пуанкаре}
\begin{definition}
	Форма $\omega$ называется точной, если существует гладкая функция $F$, такая что $\omega = dF$
\end{definition}

\begin{definition}
	Форма $\omega = F_1dx_1 + \cdots + F_mdx_m$ называется замкнутой, если
	\[\forall k,\,i:\: \frac{\partial F_i}{\partial x_k} = \frac{\partial F_k}{\partial x_i}\]
\end{definition}

\begin{definition}
	Область $\Omega \subseteq \mathbb{R}^m$ называется звёздной, если для некоторой точки $p \in \Omega$ и для любой другой точки $q \in \Omega$ отрезок $[p,\, q]$ полностью содержится в $\Omega$.
\end{definition}

\begin{lemma}
	В звёздной области любая замкнутая $C^1$-гладкая дифференциальная 1-форма точна.
\end{lemma}

\begin{proof}
	Будем считать, что точка $p$ из определения звёздной области находится в начале координат. Пусть $\omega$ -- замкнутая форма, $\omega = A_1dx_1 + \ldots A_ndx_n$.

	Заметим, что для любой точки $x = (x_1,\,\ldots,\,x_n)$ и любой функции $G:\: \mathbb{R}^n \to \mathbb{R}$
	\[G(x) - G(0) = \int_{[0,\,x]}dG = \int_0^1\left(x_1\frac{\partial G}{\partial x_1}(tx) + \cdots + x_n\frac{\partial G}{\partial x_n}(tx)\right)dt\]
	мы параметризовали отрезок $[0,\,x] \subset \mathbb{R}^n$ параметром $t$. Пользуясь этим равенством, можно восстановить любую функцию по набору её производных.

	Поэтому естественно определить $F$ таким образом:
	\[F := \int_0^1 \sum_{i = 1}^n x_iA_i(xt)dt\]
	Нам осталось проверить, что $\frac{\partial F}{\partial x_s} = A_s$. Действительно,
	\begin{align*}
		\frac{\partial F}{\partial x_s} = \int_0^1 \frac{\partial}{\partial x_s} \sum_{i = 1}^n x_iA_i(xt)dt = \int_0^1 A_s(tx) + \sum_{i = 1}^n x_i \frac{\partial}{\partial x_s}A_i(tx)dt = \\
		= \int_0^1 A_s(tx) + \sum_{i = 1}^n x_i \frac{\partial}{\partial x_i}A_s(tx)dt = \int_0^1 \frac{d}{dt}(tA_s(tx))dt = A_s(x)
	\end{align*}
	Итак, $dF = A_1dx_1 + \cdots + A_ndx_n = \omega$.
\end{proof}

\subsubsection*{Гамильтоновые векторные поля на плоскости}
\begin{definition}
	Пусть $H:\: \mathbb{R}^2 \to \mathbb{R} \in C^1(\mathbb{R}^2)$. Тогда векторное поле
	$\vec{v}: \begin{cases}
			\dot{x} = -\frac{\partial H}{\partial y}(x,\,y) \\
			\dot{y} = \frac{\partial H}{\partial x}(x,\,y)
		\end{cases}$ называется гамильтоновым тогда и только тогда, когда $\text{div }\vec{v} = 0$
\end{definition}

\subsection{Методы понижения порядка для некоторых простейших типов дифференциальных уравнений. Уравнения первого порядка, не разрешённые относительно производной.}
\subsubsection*{Методы понижения порядка дифференциальных уравнений.}
\begin{enumerate}
	\item Пусть $F(x,\, y^{(k)},\,\ldots,\,y^{(n + k)}) = 0$

	      Замена: $z = y^{(k)}$, сводим к уравнению $F(x,\, z,\, z',\,\ldots,\,z^{(n)}) = 0$
	\item Пусть $F$ явно не зависит от $x$: $F(y,\, y',\,\ldots,\,y^{(n)}) = 0$

	      Замена: $y$ -- новая независимая переменная, $y' = p = p(y)$, то есть $y''_{xx} = p'_x = p'y'=p'p$
	\item Обобщённо-однородное уравнение

	      Пусть $\exists m,\,k:\: \forall \lambda > 0:\: F(\lambda x,\, \lambda^m y,\, \lambda^{m - 1}y',\,\ldots,\,\lambda^{m - n}y^{(n)}) = \lambda^k F(x,\,y,\,y',\,\ldots,\,y^{(n)})$

	      Замена: $x = e^y,\, y = z(t)e^{mt}$
	\item Однородные уравнения

	      Пусть $\exists k:\: \forall \lambda > 0:\: F(x,\, \lambda y,\, \lambda y',\,\ldots,\, \lambda y^{(n)}) = \lambda^k F(x,\,y,\,y',\,\ldots,\,y^{(n)})$

	      Замена: $y' = z(x)y,\, y'' = (z(x)y)' = z'y + zy' = z'y + z^2y = y(z + z^2)$
\end{enumerate}

\subsubsection*{Уравнения первого порядка, не разрешённые относительно производной}
\begin{definition}
	Уравнение первого порядка, не разрешённое относительно производной -- это уравнение вида
	\[F(x,\,y,\,y') = 0\]
	где $F(x,\,y,\,y')$ -- заданная непрерывная функция в некоторой непустой окрестности $G \subseteq \mathbb{R}^3_{(x,\,y,\,p)}$ с декартовыми прямоугольными координатами $x,\,y,\,p$.
\end{definition}

\begin{note}
	В общем случае для решения уравнения применяется метод введения параметра, который позволяет свести решение исходного уравнения к решению некоторого уравнения первого порядка в симметричной форме.

	Сам метод: положим $y' = p$ и рассмотрим систему
	\begin{equation}
		\label{BASE}
		\begin{cases}
			F(x,\,y,\,p) = 0 \\
			dy = pdx
		\end{cases}
	\end{equation}
\end{note}

\begin{proposition}
	Проектирование $\pi$ поверхности $F(x,\,y,\,p) = 0$ на плоскость $(x,\,y)$ вдоль оси $p$ переводит траектории поля в интегральные кривые системы (\ref*{BASE}).

	В тех точках поверхности, где производная $\frac{\partial F}{\partial p} \neq 0$, отображение $\pi$ является локальным диффеоморфизмом.
\end{proposition}

\begin{definition}
	Точки поверхности $F(x,\,y,\,p) = 0$, в которых производная $\frac{\partial F}{\partial p} = 0$, называются особыми точками уравнения (\ref*{BASE}).
\end{definition}

\begin{definition}
	Множество всех особых точек называется криминантой, а её проекция на плоскость $(x,\,y)$ -- дискриминантной кривой уравнения (\ref*{BASE}).
\end{definition}

\begin{definition}
	Решение уравнения (\ref*{BASE}) называется особым, если его интегральная кривая является дискриминантной кривой.
\end{definition}

\begin{note}
	Решение уравнения (\ref*{BASE}) можно трактовать, как траектории движения по этой поверхности, задаваемого векторным полем
	\begin{equation}
		\label{PODN_POLE}
		\begin{cases}
			\dot{x} = \frac{\partial F}{\partial p}  \\
			\dot{y} = p\frac{\partial F}{\partial p} \\
			\dot{p} = -\left(\frac{\partial F}{\partial x} + p\frac{\partial F}{\partial y} \right)
		\end{cases}
	\end{equation}
\end{note}

\begin{definition}
	Другой важной кривой является кривая перегибов, состоящая из всех точек поверхности $F(x,\,y,\,p) = 0$, в которых третья компонента поля (\ref*{PODN_POLE}) обращается в нуль.
\end{definition}

\begin{proposition}
	Криминанта и кривая перегибов связаны некоторым двойственным соотношением: преобразование Лежандра $(x,\,y,\,p) \to (p,\, xp - y,\, x)$ переводит всякую интегральную кривую $\gamma$ уравнения (\ref*{BASE}) в интегральную кривую $\gamma^*$ сопряжённого уравнения $F(p,\, xp - y,\, x) = 0$
\end{proposition}

\section{Задача Коши}
\subsection{Прицип сжимающих отображений}
\begin{definition}
	Точка $x^* \in X$ называется неподвижной точкой отображения $\Phi$, если $\Phi(x^*) = x^*$.
\end{definition}

\begin{definition}
	Оператор $\Phi$ называется сжимающим на множестве $X$, если $\exists q \in (0,\,1)$, такое что $\forall x_1,\,x_2 \in X \mapsto \|\Phi(x_1) - \Phi(x_2)\| \leq q \|x_2 - x_1\|$.

	Число $q$ -- коэффициент сжатия.
\end{definition}

\begin{definition}
	Открытый шар $U_\varepsilon(a) = \{x \in L :\: \|x - a\| < \varepsilon\}$. Замкнутый: $\overline{U}_\varepsilon(a) = \{x \in L:\: \|x-a\|\leq\varepsilon\}$
\end{definition}

\begin{theorem}
	Теорема Банаха о неподвижной точке (принцип сижмающих отображений).

	Пусть $\Phi:\: \overline{U}_\varepsilon(x_0) \to L$, причём $\Phi$ является сжимающим на $\overline{U}_r(x_0)$ с некоторым коэффициентом $q$. Тогда если выполнено условие $\|\Phi(x_0) - x_0\| \leq (1 - q)r$, то в $\overline{U}_r(x_0)$ существует единственная неподвижная точка отображения.
\end{theorem}

\begin{proof}
	Покажем, что $\Phi(\overline{U}_r(x_0)) \subseteq \overline{U}_r(x_0)$. Пусть $x \in \overline{U}_r(x_0)$. Тогда

	\begin{align*}
		\|\Phi(x) - x_0\| = \|\Phi(x) - \Phi(x_0) + \Phi(x_0) - x_0\| \leq \|\Phi(x) - \Phi(x_0)\| + \|\Phi(x_0) - x_0\| \leq \\
		q\|x - x_0\| + (1 - q)r \leq qr + (1 - qr) = r
	\end{align*}

	Рассмотрим последовательность $\{x_n\} \subseteq \overline{U}_r(x_0)$, такую что $x_n = \Phi(x_{n - 1})$ при $n \geq 1$. Также для удобства обозначим $\rho = \|\Phi(x_0) - x_0\| = \|x_1 - x_0\|$. Покажем, что эта последовательность фундаментальная:
	\[\|x_2 - x_1\| = \|\Phi(x_1) - \Phi(x_0)\| \leq q \|x_1 - x_0\| \Rightarrow \ldots \Rightarrow \|x_{n + 1} - x_n\| \leq q^n\rho\]

	Используем полученную оценку для того, чтобы оценить модули в сумме:
	\[\forall p:\: \|x_{n + p} - x_n\| \leq \sum_{i = 1}^p \|x_{n + i} - x_{n + i - 1}\| \leq \rho \sum_{i = 1}^p q^{n + i - 1} = \frac{\rho q^n(1 - q^p)}{1 - q} < \frac{\rho q^n}{1 - q} \to 0\]

	А так как мы в банаховом пространстве, то из фундаментальности получили сходящуюся последовательность, т.е. $\exists x^* = \lim\limits_{n \to \infty} x_n$. А так как $\overline{U}_r(x_0)$ -- замкнутый шар, значит $x^* \in \overline{U}_r(x_0)$.

	Докажем, что $x^*$ является неподвижной точкой оператора $\Phi$. Воспользуемся тем, что сжимающее отображение является непрерывным.

	В $x_n = \Phi(x_{n - 1})$ перейдём к пределу: $x^* = \lim\limits_{n \to \infty} = \Phi(\lim\limits_{n \to \infty} x_{n - 1}) = \Phi(x^*)$.

	Докажем единственность неподвижной точки. Допустим, что существует $x^{**} \in \overline{U}_r(x_0):\: x^{**} = \Phi(x^{**})$, такое, что $x^{**} \neq x^*$.

	Тогда $\|x^* - x^{**}\| = \|\Phi(x^*) - \Phi(x^**)\| \leq q\|x^* - x^{**}\|$, где $q < 1$. Получили противоречие.
\end{proof}

\subsection{Теоремы существования и единственности решения задачи Коши}
\begin{definition}
	Пусть $n \geq 2,\, f_1,\, \ldots,\, f_n$ -- непрерывные функции, определённые на $G \subseteq \mathbb{R}^{n + 1}_{(x,\, \vec{y})}$. Назовём нормальной системой дифференциальных уравнений первого порядка следующую систему:
	\begin{equation}
		\label{NORMAL_SYS}
		\begin{cases}
			y'_1(x) = f_1(x,\,y_1(x),\,\ldots,\,y_n(x)) = f_1(x,\,\vec{y}) \\
			\vdots                                                         \\
			y_n'(x) = f_n(x,\,y_1(x),\,\ldots,\,y_n(x)) = f_n(x,\,\vec{y})
		\end{cases}
	\end{equation}

\end{definition}

\begin{definition}
	Вектор-функция $\vec{\phi}(x)$ называется решением нормальной системы (\ref*{NORMAL_SYS}) на некотором промежутке $I \subseteq \mathbb{R}$, если:
	\begin{enumerate}
		\item $\vec{\phi}(x) \in C^1(I)$
		\item $\forall x \in I:\: (x,\,\vec{\phi}(x)) \in G$
		\item $\forall x \in I:\: \vec{\phi'}(x) = \vec{f}(x,\, \vec{\phi}(x))$
	\end{enumerate}

	График решения $\vec{\phi}(x)$ в пространстве $\mathbb{R}^{n + 1}$ -- это интегральная кривая
\end{definition}

\begin{definition}
	Задача Коши -- это

	\[\begin{cases}
			\vec{y'} = \vec{f}(x,\, \vec{y}) \\
			\vec{y}(x_0) = \vec{y}_0
		\end{cases}\]
\end{definition}

\begin{definition}
	Вектор-функция $\vec{f}(x,\,\vec{y})$, определённая в области $G \subseteq \mathbb{R}^{n + 1}$ называется удовлетворяющей условию Липшица относительно $\vec{y}$ равномерно по $x$, если $\exists L > 0:\: \forall (x,\, \vec{y_1}),\, (x,\,\vec{y_2}):\: |\vec{f}(x,\,\vec{y_1}) - \vec{f}(x,\,\vec{y_2})| \leq L|\vec{y_1} - \vec{y_2}|$.
\end{definition}

\begin{lemma}
	Вектор-функция $\vec{f}(x,\,\vec{y})$ удовлетворяет условию Липшица по $\vec{y}$ равномерно по $x$ при выполнении следующих условий:

	\begin{enumerate}
		\item $G$ -- выпуклая область в $\mathbb{R}^{n + 1}$
		\item $\vec{f}(x,\,\vec{y}) \in C_n(G)$, то есть непрерывна от $n$ аргументов и $\forall i,\, j = \overline{1,\,n}:\: \frac{\partial f_i}{\partial y_j} \in C(G)$
		\item $\exists K > 0:\: \forall i,\,j = \overline{1,\,n}:\: \forall (x,\, \vec{y}) \in G:\: \|\frac{\partial \vec{f_i}}{\partial y_j}(x,\,\vec{y})\| \leq K$
	\end{enumerate}
\end{lemma}

\begin{proof}
	Фиксируем $i = 1,\,\ldots,\,n$. Рассмотрим $(x,\, \vec{y_1})$, где $\vec{y_1} = (y_1^1,\,\ldots,\,y_n^1)$, а также $\vec{y_2} = (y_1^2,\,\ldots,\,y_n^2)$.

	\begin{align*}
		|f_i(x,\,\vec{y_1}) - f_i(x,\,\vec{y_2})| = \||f_i(x,\, \vec{y_2} + \theta(\vec{y_1} - \vec{y_2})|_{\theta = 1}^{\theta = 2}\| = \|\int_0^1 \left[\frac{d}{d\theta}f_i(x,\,\vec{y_2} + \theta(\vec{y_1} - \vec{y_2}))\right]d\theta\| = \\
		\|\int_0^1 \sum_{j = 1}^n \frac{\partial f_i(x,\, \vec{y_2} + \theta(\vec{y_1} - \vec{y_2}))}{\partial y_j}(y^1_j - y^2_j)d\theta\| \leq n \cdot K \cdot |\vec{y_1} - \vec{y_2}|
	\end{align*}
\end{proof}

\subsubsection*{Теорема о существовании и единственности решения задачи Коши для системы уравнений $n$-го порядка в нормальной форме}

\begin{theorem}
	Пусть вектор-функция $\vec{f}(x,\, \vec{y})$ непрерывна в области $G$ вместе со своими производными по $y_j (j = \overline{1,\,n})$, точка $(x_0,\, \vec{y_0})$ тоже лежит в $G$. Тогда задача Коши локально разрешима единственным образом:
	\begin{enumerate}
		\item $\exists \delta > 0$, такое что на $[x_0 - \delta,\, x_0 + \delta]$ решение задачи Коши существует.
		\item Решение единственно в следующем смысле:

		      Если $\vec{y_1} \equiv \vec{\phi}(x)$ -- решение задачи Коши в $\delta_1$-окрестности точки $x_0$, а $\vec{y_2} \equiv \vec{\psi}(x)$ -- решение задачи Коши в $\delta_2$-окрестности точки $x_0$, то в окрестности точки $x_0$ с радиусом $\delta = \min(\delta_1,\,\delta_2):\: \vec{\phi}(x) \equiv \vec{\psi}(x)$
	\end{enumerate}
\end{theorem}

\begin{proof}
	Рассмотрим множество $\overline{H_{\delta,\,r}}(x_0,\, \vec{y_0}) = \{(x,\,\vec{y}) \in G:\: x \in [x_0 - \delta,\, x_0 + \delta] \land \|\vec{y} - \vec{y_0}\| \leq r\}$. Заметим, что в силу компактности этого множества применима теорема Вейерштрасса:
	\[\exists M > 0:\: \forall (x,\,\vec{y}) \in \overline{H_{\delta,\,r}}:\: |\vec{f}(x,\,\vec{y})| \leq M,\, \forall i,\,j = \overline{1,\,n} \: |\frac{\partial f_i}{\partial y_j}| \leq M\]
	Рассмотрим интегральное уравнение
	\[\vec{y}(x) = \vec{y_0} + \int_{x_0}^x \vec{f}(t,\, \vec{y}(t))dt \Leftrightarrow \vec{y} = \Phi(\vec{y})\]
	Рассмотрим в $C_n[x_0 - \delta,\, x_0 + \delta]$ замкнутый шар $\overline{D_{\delta,\,r}}(\vec{y_0}) = \{\vec{y} \in C_n[x_0 - \delta,\, x_0 + \delta]:\: \|\vec{y} - \vec{y_0}\|_{C_n} \leq r\}$, где $\|\vec{y}\|_{C_n} = \max\limits_{1 \leq i \leq n} \sup\limits_{|x - x_0| < \delta}|y_i(x)|$.

	Докажем, что существуют $\delta$ и $r$ такие, что
	\begin{itemize}
		\item $\Phi$ является сжимающим
		\item Отображает шар $\overline{D_{\delta,\,r}}$ в себя
	\end{itemize}
	Тогда мы сможем применить теорему Банаха о сжимающем отображении. Получим единственную неподвижную точку отображения $\Leftrightarrow$ интегральное уравнение имеет единственное решение $\Leftrightarrow$ задача Коши имеет единственное решение.

	Докажем, что $\Phi$ является сжимающим. Рассмотрим $\vec{y},\, \vec{z} \in \overline{D_{\delta,\,r}}$:
	\begin{align*}
		\|\Phi(\vec{y}) - \Phi(\vec{z})\| = \max_{1 \leq i \leq n} \sup_{|x - x_0| < \delta} |\int_{x_0}^x (f(\tau,\, \vec{y}(\tau)) - f(\tau,\, \vec{z}(\tau)))|\tau \leq  \sup_{|x - x_0| < \delta} \int_{x_0}^x L|\vec{y}(\tau) - \vec{z}(\tau)|d\tau \leq \\
		\sup_{|x - x_0| < \delta} \int_{x_0}^x L\|\vec{y} - \vec{z}\|_{C_n} d\tau \leq \delta L \|\vec{y} - \vec{z}\|_{C_n}
	\end{align*}
	Положив $\delta = \frac{q}{L}$, получим требуемое.

	Теперь докажем вторую часть:
	\begin{align*}
		\|\Phi(\vec{y_0}) - \vec{y_0}\| = \max_{1 \leq i \leq n} \sup_{|x - x_0| < \delta} |\int_{x_0}^x f_i(\tau,\, \vec{y_0})d\tau|\leq \int_{x_0}^x \|\vec{f}(\tau,\, \vec{y_0})\|_{C_n}d\tau \leq \delta M := (1 - q)r
	\end{align*}
	Получили, что
	\[
		\begin{cases}
			q = \delta L \\
			(1 - q)r = \delta M
		\end{cases}
		\Rightarrow
		\begin{cases}
			r - rq = \delta \\
			r = \delta Lr + \delta M
		\end{cases}
		\Rightarrow
		\delta_r := \frac{r}{M + Lr}
	\]
\end{proof}

\begin{definition}
	Нормальный вид уравнения с ЗК, разрешённого относительно старшей производной выглядит так:
	\[
		\begin{cases}
			y^{(n)} = f(x,\,y,\,y',\,\ldots,\,y^{(n-1)}) \\
			y(x_0) = y_0                                 \\
			y'(x_0) = y'_0                               \\
			\vdots                                       \\
			y^{(n - 1)}(x_0) = y^{(n-1)}_0
		\end{cases}
	\]
\end{definition}

\subsubsection*{Теорема о существовании и единственности решения задачи Коши для уравнения $n$-го порядка в нормальном виде}
\begin{theorem}
	Пусть функция $f(x,\,y,\,p_1,\,\ldots,\,p_{n-1})$ определена и непрерывна по совокупности переменных вместе с частными производными по переменным $y,\,p_1,\,\ldots,\,p_{n - 1}$ в некоторой области $G \subseteq \mathbb{R}^{n - 1}$, и точка $(x_0,\,y_0,\,y_0',\,\ldots,\,y^{(n-1)}_0) \in G$. Тогда существует замкнутая $\delta$-окрестность точки $x_0$, в которой существует единственное решение задачи Коши.
\end{theorem}

\subsection{Теорема о продолжении решения нормальной системы дифференциальных уравнений}
\begin{theorem}
	Пусть $\vec{f}(x,\,\vec{y})$ определена и удовлетворяет условиям теоремы о существовании и единственности решения задачи Коши на замыкании ограниченной области $G \subseteq \mathbb{R}^{n + 1}$, тогда любое решение задачи Коши
	\[
		\begin{cases}
			\vec{y'} = \vec{f}(x,\,\vec{y}) \\
			\vec{y}(x_0) = \vec{y_0}
		\end{cases}
	\]
	можно продолжить в обе стороны до выхода $\Gamma = \partial G$, то есть можно доопределить $\vec{y}(x)$ на некоторый $[x_0 - \delta,\,x_0 + \delta] \subseteq [a,\,b]$, причём $(a,\,\vec{y}(a)),\, (b,\,\vec{y}(b)) \in \Gamma$
\end{theorem}

\begin{proof}
	\[\overline{H_r}(x_0,\, \vec{y_0}) := \{(x,\,\vec{y}) \in G:\: x \in [x_0 - \delta_r,\, x_0 + \delta_r] \land \|\vec{y} - \vec{y_0}\| \leq r\} \subseteq G\]
	\[\rho((x_1,\,\vec{y_1}),\, (x_2,\,\vec{y_2})) := \max(|x_1 - x_2|,\, \|\vec{y_1} - \vec{y_2}\|)\]
	Введём также расстояние от точки $\rho$ до множества $M:\: \rho(p,\, M) = \inf\limits_{q \in M} \rho(p,\,q)$

	Наконец, определим $\delta_0,\, r_0$ как значения $\delta_r$ и $r$ для точки $p_0 = (x_0,\,\vec{y_0})$, для которых выполнено $\max(\delta_0,\, r_0) = \rho(p_0,\, \Gamma)$.

	Рассмотрим $x_1 = x_0 + \delta_0$ и $\vec{y_1} = \vec{y}(x_1)$. Обозначим $p_1 = (x_1,\, \vec{y_1})$
	\begin{itemize}
		\item Если $p_1 \in \Gamma$, то продолжение вправо не требуется
		\item Если $p_1 \not\in \Gamma$, то $p_1$ -- внутренняя точка, а значит, в ней $\exists$! решение ЗК, причём оно совпадает с решением для $p_0$ на $[x_0,\, x_0 + \delta_0] \cap [x_1 - \delta_1,\, x_1]$. Аналогично определяем $p_2$ и т.д.
	\end{itemize}

	Полученная последовательность $\{p_k\}$ монотонно возрастает по $x$ и ограничена точками из $\Gamma$. Следовательно, по теореме Вейшерштрасса существует $b = \lim\limits_{k \ to \infty} x_k = x_0 + \sum\limits_{k = 0}^\infty \delta_k$. Объединение решений задач Коши является функцией, определённой на $\bigcup\limits_{k = 1}^\infty [x_0,\,x_k] = [x_0,\, b)$. Зафиксируем $\varepsilon > 0$ и рассмотрим $\alpha,\, \beta \in [b - \varepsilon,\, b)$. Заметим, что
	\[\|\vec{y}(\beta) - \vec{y}(\alpha)\| = \|\int_\alpha^\beta \vec{f}(\tau,\, \vec{y}(\tau))d\tau\| \leq \int_\alpha^\beta |\vec{f}(\tau,\, \vec{y}(\tau))d\tau| \leq M\varepsilon\]
	Значит, по критерию Коши существует $y^* = \lim\limits_{x \to b - 0} \vec{y}(x)$. Пусть $p^* = (b,\, \vec{y^*})$.
	\[0 \leq \rho(p_k,\, \Gamma) = \max(\delta_k,\, r_k) \to 0\]
	Значит $\rho(p^*,\, \Gamma) = \lim\limits_{k \to \infty}\rho(p_k,\, \Gamma) = 0 \Rightarrow p^* \in \Gamma$
	\[\frac{\vec{y}(b) - \vec{y}(b - \varepsilon)}{\varepsilon} = \int_{b - \varepsilon}^b \frac{\vec{f}(\tau,\, \vec{y}(\tau))}{\varepsilon}d\tau = \frac{\vec{f}(\xi,\, \vec{y}(\xi)) \cdot \varepsilon}{\varepsilon} \to \vec{f}(b,\,\vec{y}(b))\]
	$\vec{f}$ непрерывна вплоть до $\Gamma \Rightarrow f \in C(p^*)$. Тогда можем сделать второй переход по интегральной теореме о среднем, где $\xi \in [b - \varepsilon,\, b]$.

	Таким образом, мы успешно продлили вправо, аналогично можно продлить и влево.
\end{proof}

\subsection{Непрерывная зависимость от параметров решения задачи Коши для нормальной системы дифференциальных уравнений}
\begin{note}
	Дифференциальные уравнения, описывающие физические процессы, всегда содержат некоторые параметры. Эти параметры в реальных задачах никогда не могут быть измерены абсолютно точно, то есть всегда измеряются с некоторой погрешностью, так что сами дифференциальные уравнения известны лишь с некоторой степенью точности. Поэтому, для того чтобы уравнения могли описывать реальные процессы, необходимо, чтобы их решения непрерывно зависели от параметров, то есть чтобы они мало менялись при малых изменениях параметров.
\end{note}

Рассмотрим задачу Коши для нормальной системы дифференциальных уравнений в векторном виде, то есть $\vec{f} = (f_1,\,\ldots,\,f_n)$:
\begin{equation}\label{PARAM_CAUCHY}
	\frac{d\vec{y}}{dx} = \vec{f}(x,\,\vec{y},\,\vec{\mu})\;\;\; \vec{y}(x_0,\, \vec{\mu}) = \vec{y_0}(\vec{\mu})
\end{equation}

\begin{theorem}
	Пусть $\vec{f}(x,\,\vec{y},\,\vec{\mu})$ -- непрерывна и удовлетворяет условию Липшица равномерно по $x$ и $\vec{\mu}$, $\forall (x,\,\vec{y}) \in G \subseteq \mathbb{R}^{n + 1}$ и всех $\vec{\mu}$, таких, что $|\vec{\mu} - \vec{\mu_0}| \leq \delta$. Пусть, кроме того, $(x_0,\, \vec{y_0}) \in G$.

	Тогда $\exists h > 0 \mapsto$ решение $\vec{y}(x,\,\vec{\mu})$ задачи Коши непрерывно по совокупности переменных $(x,\,\vec{\mu})$ в некоторой области $|x - x_0| \leq h,\, |\vec{\mu} - \vec{\mu_0}| \leq \delta$.

	\subsection{Дифференцируемость и гладкость решения по параметрам, уравнение в вариациях}

	\begin{lemma}
		Лемма Адамара.

		Пусть $F(x,\,u):\: \mathbb{R}^n_x \times \mathbb{R}^m_u \to \mathbb{R}^1$. $F \in C^p(D),\, p \geq 1,\, D$ -- область в $\mathbb{R}^{n + m}$. Тогда $\exists H_i(x,\,u,\,z) \in C^{p - 1}(D),\, i=\overline{1,\,m}$:
		\[F(x,\,\hat{u}) - F(x,\,u) = \sum_{i = 1}^m H_i(x,\,u,\,\hat{u})\cdot(\hat{u_i} - u_i)\]
	\end{lemma}

	\begin{theorem}
		Если при $(x,\,y) \in G$ и $|\vec{\mu} - \vec{\mu_0}| \leq \delta$ функции
		\[f(x,\,y,\,\vec{\mu})\;\;\; \frac{\partial f}{\partial y}\;\;\; \frac{\partial f}{\partial \mu_i}\]
		непрерывны, а также $(x_0,\,y_0) \in G$, то $\exists h > 0$, что при $|x - x_0| \leq h,\, |\vec{\mu} - \vec{\mu_0}| \leq \delta$ для решения $y = \phi(x,\, \vec{\mu})$ задачи Коши с параметров верно следующее:
		\begin{enumerate}
			\item $z_i(x,\,\vec{\mu}) = \frac{\partial \phi}{\partial \mu_i}$ непрерывны для указанных $x$ и $\vec{\mu}$.
			\item Смешанные производные $\frac{\partial^2 \phi}{\partial x \partial \mu_i}$ непрерывны и не зависят от порядка дифференцирования.
			\item Частные производные $z_i$ удовлетворяют уравнениям в вариациях по параметру $\vec{\mu}$:
			      \[\frac{\partial z_i}{\partial x} = \frac{\partial f(x,\, \phi(x,\,\vec{\mu}),\,\vec{\mu})}{\partial x}\cdot z_i + \frac{\partial f(x,\, \phi(x,\,\vec{\mu}),\,\vec{\mu})}{\partial \mu_i}\]
			      и начальным условиям $z_i(x_0,\,\vec{\mu}) = 0$
		\end{enumerate}
	\end{theorem}

	\begin{proof}
		Для $\mu \in \mathbb{R}^1$.

		$\exists \frac{\partial \phi}{\partial \mu}:\: \phi(x,\,\mu)$ -- решение (\ref*{PARAM_CAUCHY}), $\phi(x,\, \mu + \Delta\mu)$ -- решение (\ref*{PARAM_CAUCHY}), где $\mu \mapsto \mu + \Delta\mu$. Введём $\psi(x,\,\mu,\,\Delta\mu) := \frac{\phi(x,\, \mu + \Delta\mu) - \phi(x,\, \mu)}{\Delta\mu},\, \forall \Delta\mu \neq 0$.

		Тогда
		\begin{align*}
			\psi' = \frac{1}{\Delta\mu}\left[f(x,\, \phi(x,\, \mu),\, \mu + \Delta\mu) - f(x,\,\phi(x,\, \mu),\, \mu)\right] \overset{\text{л. Ада.}}{=}            \\
			\frac{1}{\Delta\mu}\left[\sum_{i=1}^n h_i(x,\, u,\,\hat{u})(\phi(x,\, \mu + \Delta\mu) - \phi(x,\, \mu)) + h_{n + 1}(x,\,u,\,\hat{u})\Delta\mu\right] = \\
			\sum_{i = 1}^n h_i(x,\,u,\,\hat{u})\psi(x,\,\mu,\,\Delta\mu) + h_{n + 1}(x,\,u,\,\hat{u})
		\end{align*}
		Таким образом,
		\begin{equation}\label{VARI_EQ}
			\begin{cases}
				\psi' = H_1(x,\,\mu,\,\Delta\mu)\cdot\psi + \hat{H_2}(x,\,\mu,\,\Delta\mu) \\
				\psi(x_0) = 0
			\end{cases}
		\end{equation}
		Значит $\exists! \psi(x,\,\mu,\, \Delta\mu) = \frac{\partial \phi}{\partial \mu} \in C$
	\end{proof}
\end{theorem}

\begin{definition}
	Уравнение (\ref*{VARI_EQ}) называют уравнением в вариациях для решения $\phi$.
\end{definition}

\section{Линейные дифференциальные уравнения и линейные системы дифференциальных уравнений с постоянными коэффициентами}
\subsection{Фундаментальная система решений и общее решение линейного однородного уравнения n-го порядка}
Факты ниже перекликаются со следующим разделом
\begin{definition}
	Вектор-функция $\vec{y_1}(x),\,\ldots,\,\vec{y_k}(x)$, определённые на промежутке $I$, называются линейно зависимыми, если
	\[\exists \alpha_1,\,\ldots,\,\alpha_k \in \mathbb{R}:\: \exists i:\: \alpha_i \neq 0:\: \sum_{j = 1}^k \alpha_j \vec{y_j}(x) \equiv 0\]
\end{definition}

\begin{definition}
	Пусть $\vec{y_1}(x),\,\ldots,\,\vec{y_n}(x)$ -- вектор-функции с $n$ компонентнами. Функция
	\[W(x) = \begin{vmatrix}
			y_1^1(x) & y_2^1(x) & \ldots & y_n^1  \\
			\vdots   & \vdots   & \ddots & \vdots \\
			y_1^n(x) & y_2^n(x) & \ldots & y_n^n  \\
		\end{vmatrix}\]
	называется определителем Вронского для заданных вектор-функций.
\end{definition}

\begin{lemma}
	Если вронскиан системы $\vec{y_1}(x),\,\ldots,\,\vec{y_n}(x)$ отличен от нуля хотя бы в одной точке, то все эти функции линейно независимы.
\end{lemma}

\begin{proof}
	Пусть эти функции линейно зависимы, тогда векторы $\vec{y_1}(x),\,\ldots,\,\vec{y_n}(x)$ линейно зависимы в каждой точке $x_0$, а значит, определитель матрицы, составленной из векторов, равен нулю.
\end{proof}

\begin{lemma}
	Если вектор-функции $\vec{y_1}(x),\,\ldots,\,\vec{y_n}(x)$ -- решения некоторой системы линейных однородных уравнений на промежутке $I$ и $\exists x_0 \in I:\: W(x_0) = 0$, то $\vec{y_1}(x),\,\ldots,\,\vec{y_n}(x)$ линейно зависимы на $I$.
\end{lemma}

\begin{proof}
	В точке $x_0$ векторы $\vec{y_1}(x),\,\ldots,\,\vec{y_n}(x)$ линейно зависимы, значит, существует их линейная комбинация, которая обращается в ноль в точке $x_0:\: \sum\limits_{i = 1}^n c_i\vec{y_i}(x_0) = 0$.

	Рассмотрим $n$ задач Коши для $\vec{y}(x_0) = \vec{y_i}(x_0)$. Рассмотрим функцию $\vec{y}(x) = \sum\limits_{i = 1}^n c_i \vec{y_i}(x)$ на $I$.

	$\vec{y}(x_0) = 0 \Rightarrow$ по теореме о существовании и единственности $\vec{y}(x) \equiv 0$.
\end{proof}

\begin{definition}
	Фундаментальная система решений для СЛДУ -- набор $n$ линейно независимых решений системы.
\end{definition}

\begin{lemma}
	Для любой СЛДУ существует ФСР.
\end{lemma}

\begin{proof}
	Пусть $x_0 \in [a,\,b]$ и $\{\vec{y_0^i}\}_{i = 1}^n$ -- набор числовых линейно независимых векторов. Для каждого из числовых векторов составим соответствующую задачу Коши и получим $\vec{z_1},\,\ldots,\,\vec{z_n}$ -- решения этих задач, тогда их вронскиан равен определителю матрицы, составленной из $\{\vec{y_0^i}\}_{i = 1}^n$, следовательно, он не равен нулю, и ФСР существует.
\end{proof}

\begin{lemma}
	Любое решение СЛДУ единственным образом представимо в виде линейной комбинации векторов ФСР.
\end{lemma}

\begin{proof}
	Пусть $x_0 \in [a,\,b],\, \vec{y}$ -- решение системы, и $\vec{y_1},\,\ldots,\,\vec{y_n}$ -- ФСР, тогда

	$\vec{y_1}(x_0),\,\ldots,\,\vec{y_n}(x_0)$ линейно независимы, и $\vec{y}(x_0)$ единственным образом выражается через них. В силу единственности решения задачи Коши коэффициенты линейной комбинации окажутся одними и теми же для всех точек отрезка.
\end{proof}

\subsubsection*{Линейные однородные ДУ $n$-го порядка с постоянными коэффициентами}
\begin{definition}
	Линейным однородным ДУ $n$-го порядка с постоянными коэффициентами называется уравнение вида
	\[L(y) = a_0y^{(n)} + a_1y^{(n - 1)} + \cdots + a_{n - 1}y' + a_ny = 0\]
\end{definition}

\begin{note}
	Введём $\vec{y} = (y,\, y',\,\ldots,\,y^{(n-1)})$. Используя результаты уравнения $L(y)$ и тем, что $\vec{y'_i} = \vec{y_{i+1}}$ получим систему:
	\[\begin{cases}
			\vec{y_1'} = \vec{y_2} \\
			\vec{y_2'} = \vec{y_3} \\
			\ldots                 \\
			\vec{y_n'} = \frac{-1}{a_0}(a_n\vec{y_1} + a_{n-1}\vec{y_2} + \cdots + a_1\vec{y_n})
		\end{cases}\]

	После введения данной системы существование и единственность решения исходного уравнения очевидна.

	Для исходного уравнения можно определить вронскиан, равный вронскиану написанной выше системы.
\end{note}

\begin{note}
	Найдём решение системы 1 в виде $y = e^{\lambda x}$
	\[M(\lambda) = a_0\lambda^n + \cdots + a_{n-1}\lambda + a_n\]
	\[L(e^{\lambda x}) = M(\Lambda)e^{\lambda x} = 0\]
	Поскольку экспонента никогда не обнуляется, то единственный возможный вариант -- это $M(\lambda) = 0$. Уравнение $M(\lambda) = 0$ называется характеристическим, как и многочлен в его в левой части.
\end{note}

\begin{proposition}
	Пусть $\lambda_1,\,\ldots,\,\lambda_n$ -- однократные корни $M(\lambda)$, тогда решения $y_i = e^{\lambda_i x}$ линейно независимы.
\end{proposition}

\begin{proof}
	\[W(e^{\lambda_1 x},\,\ldots,\,e^{\lambda_n x}) = \begin{vmatrix}
			e^{\lambda_1 x}                & \ldots & e^{\lambda_n x}                \\
			\lambda_1e^{\lambda_1 x}       & \ldots & \lambda_ne^{\lambda_n x}       \\
			\vdots                         & \ddots & \vdots                         \\
			\lambda_1^{n-1}e^{\lambda_1 x} & \ldots & \lambda_n^{n-1}e^{\lambda_n x}
		\end{vmatrix} = e^{(\lambda_1 + \cdots + \lambda_n)x}\begin{vmatrix}
			1               & \ldots & 1               \\
			\lambda_1       & \ldots & \lambda_n       \\
			\vdots          & \ddots & \vdots          \\
			\lambda_1^{n-1} & \ldots & \lambda_n^{n-1}
		\end{vmatrix}\]
	Получившийся определитель Вандермонда не будет равен нулю в силу того, что все $\lambda_i$ различны.

	Если все $a_i \in \mathbb{R}$, то все комплексные корни $M(\lambda)$ разбиваются на пары сопряжённых между собой комплексных чисел. Мнимая и действительная части решений, соответствующих таким корням, сами являются решениями:
	\[\lambda = \alpha + i\beta\;\;\; \overline{\lambda} = \alpha - i\beta\]
	\[y_1 = e^{\lambda x}\;\;\; \overline{y_1} = e^{\overline{\lambda}x}\]
	\[z_1 = \frac{y_1 + \overline{y_1}}{2} = e^{\alpha x}\cos \beta x\;\;\; z_2 = \frac{y_1 - \overline{y_1}}{2} = e^{\alpha x}\sin \beta x\]
	Замена $y_1,\,\overline{y_1}$ на $z_1,\,z_2$ является корректным переходом в другой базис, а потому линейная оболочка не изменится.
\end{proof}

\begin{proposition}
	Пусть $\lambda$ -- корень $M(\lambda)$ кратности $l$, тогда функции
	\[e^{\lambda x},\, xe^{\lambda x},\,\ldots,\, x^{l - 1}e^{\lambda x}\] являются решениями.
\end{proposition}

\begin{lemma}
	Пусть $y = x^se^{\lambda x}$, где $\lambda$ -- корень характеристического уравнения кратности $l$, тогда
	\[L(x^se^{\lambda x}) = \begin{cases}
			0,\, s < l \\
			(b_0x^{s-l} + b_1x^{s-l-1}+\cdots+b_{s-l})e^{\lambda x},\, x \geq l
		\end{cases}\]

	\begin{proof}
		Пусть $z,\, \lambda$ -- комплексные числа. Сначала докажем, что $(z^se^{\lambda z})^{(p)}_z = (\lambda^pe^{\lambda z})^{(s)}_\lambda$. Здесь мы используем операцию формального дифференцирования.

		Докажем наше утверждение:
		\begin{align*}
			(z^se^{\lambda z})^{(p)}_z = \sum_{k=0}^p C_p^k (z^s)_z^{(k)}(e^{\lambda z})_z^{(p - k)} = \sum_{k = 0}^p C_p^k s(s-1)\ldots(s - (k - 1))z^{s - k}\lambda^{p-k}e^{\lambda z} = \\
			\sum_{k = 0}^{\min(p,\,s)} C_p^k C_s^k k! z^{s - k}\lambda^{p-k}e^{\lambda z}
		\end{align*}
		Заметич, что $(\lambda^pe^{\lambda z})^{(s)}_\lambda$ раскроется в такое же выражение ввиду структурной симметрии.

		Найдём:
		\begin{align*}
			L(x^se^{\lambda x}) = a_0(x^se^{\lambda x})^{(n)}_x + a_1(x^se^{\lambda x})^{(n-1)}_x + \cdots + a_n(x^se^{\lambda x}) =                                                                             \\
			a_0 ((e^{\lambda x})^{(s)}_\lambda)^{(n)}_x + \cdots + a_n (e^{\lambda x})^{(s)}_\lambda = a_0((e^{\lambda x})^{(n)}_x)^{(s)}_\lambda + \cdots + a_n(e^{\lambda x})^{(s)}_\lambda =                  \\
			(a_0(e^{\lambda x})^{(n)}_x + \cdots + a_ne^{\lambda x})^{(s)}_\lambda = (e^{\lambda x}M(\lambda))^{(s)}_\lambda = \sum_{k = 0}^s C_s^k(M(\lambda))_\lambda^{(k)}(e^{\lambda x})^{(s - k)}_\lambda = \\
			\sum_{k = l}^s C_s^k(M(\lambda))_\lambda^{(k)}(e^{\lambda x})^{(s - k)}_\lambda
		\end{align*}
		Следовательно, $b_i = C_s^k(M(\lambda))^{(k)}_\lambda(e^{\lambda x})^{(s-k)}_\lambda$

		Исходное утверждение выводится из леммы применением того факта, что корень кратности $s$ многочлена $P(x)$ является корнем $P,\,P',\,\cdots,\,P^{(s-1)}$
	\end{proof}
\end{lemma}

\begin{definition}
	Квазимногочлен -- произведение многочлена на экспоненту с линейной функцией в показателе.
\end{definition}

\begin{proposition}
	$(P_m(x)e^{\lambda x})'_x = Q_m(x)e^{\lambda x}$
\end{proposition}

\begin{theorem}
	О структуры ФСР.

	Пусть $\lambda_1,\,\cdots,\, \lambda_k$ корни характеристического многочлена $M(\lambda)$ кратности $l_1,\,\cdots,\,l_k$. Тогда набор функций $x^se^{\lambda_i x}$, где $s = 0,\,\cdots,\,l_i - 1,\, i = 1,\,\cdots,\,k$ является ФСР для исходного уравнения.
\end{theorem}

\begin{proof}
	Докажем от противного. Пусть ЛЗ, то есть
	\[\exists \{c_i\}_{i = 1}^n \neq 0:\: \sum_{i = 1}^n c_iy_i(x) \equiv 0\]
	Сгруппируем слагаемые при одинаковых $e^{\lambda_i x}:\: \sum\limits_{i = 1}^k e^{\lambda_i x}p_i(x) \equiv 0$. Значит, хотя бы один из многочленов $p_i(x) \neq 0$. Б.О.О. примем $p_1(x) \neq 0$, домножим равенство на $e^{-\lambda_k x}$:
	\[e^{x(\lambda_1 - \lambda_k)}p_1(x) + e^{x(\lambda_2 - \lambda_k)}p_2(x) + \cdots + p_k(x) = 0\]
	Продифференцируем $l_k$ раз:
	\[e^{x(\lambda_1 - \lambda_k)}Q_1(x) + e^{x(\lambda_2 - \lambda_k)}Q_2(x) + \cdots + e^{x(\lambda_{k-1} - \lambda_k)}Q_{k-1}(x) = 0\]
	Разделим на $e^{x(\lambda_{k-1} - \lambda_k}$ и будем повторять процедуру, пока не дойдём до
	\[e^{x(\lambda_1 - \lambda_2)}R(x) = 0\]
	Откуда $R(x) = 0$, но тогда $p_1(x) = 0$. Противоречие.
\end{proof}

\subsection{Линейное неоднородное уравнение n-го порядка с постоянными коэффициентами и правой частью квазимногочленом}
\begin{definition}
	Эти уравнения имеют вид
	\[y^{(n)} + a_1y^{(n-1)} + \cdots + a_ny = f(x)\]
	где $f(x)$ квазимногочлен: $f(x) = e^{\mu x}P_m(x),\, \mu \in \mathbb{C},\, P_m(x)$ -- заданный многочлен степени $m$ с комплексными коэффициентами.
\end{definition}

\begin{definition}
	Характеристическим многочленом $L(x)$ назовём многочлен
	\[L(x) = a_nx^n + a_{n-1}x^{n-1} + \cdots + a_0\]
\end{definition}

\begin{note}
	Аналогично однородному случаю, существование и единственность следуют из таковых для системы
	\[\begin{cases}
			\vec{y_1'} = \vec{y_2}     \\
			\vec{y_2'} = \vec{y_3}     \\
			\cdots                     \\
			\vec{y_{n-1}'} = \vec{y_n} \\
			\vec{y_n'} = f - a_1y_1 - a_2y_2 - \cdots - a_ny_n
		\end{cases}\]
\end{note}

\begin{definition}
	Если число $\mu$ является корнем характеристического уравнения
	\[L(\lambda) = 0\]
	то говорят, что в уравнении резонансный случай. Если же $\mu$ не является корнем, то имеем нерезонансный случай.
\end{definition}

\begin{definition}
	Дифференциальным многочленом назовём многочлен вида
	\[L(D) = (D - \lambda_1)^{k_1}(D - \lambda_2)^{k_2}\cdots(D - \lambda_s)^{k_s}\]
	где $k_i$ соответствует кратности корней характеристического уравнения, а $D$ -- оператор формального дифференцирования.
\end{definition}

\begin{note}
	Рассмотрим ЛОУ. Покажем, что если известно некоторое решение $y_0(x)$ ЛНУ, то замена $y = z + y_0$ приводит уравнение к ЛОУ. Воспользуемся представлением левой части через дифференциальный многочлен:
	\[L(D)y = L(D)(z + y_0) = L(D)z + L(D)y_0 = L(D)z + f(x) = f(x)\]
	Отсюда следует, что $L(D)z = 0$, то есть решение.
\end{note}

Рассмотрим $L(D)y(x) = e^{\mu x}P_m(x)$
\begin{proposition}
	$(P_m(x)e^{\lambda x})'_x = Q_m(x)e^{\lambda x}$
\end{proposition}

\begin{theorem}
	О структуре решения ЛНУ с правой частью в виде квазимногочлена.

	Для рассматриваемого уравнения существует и единственно решение вида
	\[y(x) = x^ke^{\mu x}Q_m(x)\]
	где $Q_m(x)$ -- многочлен одинаковой с $P_m(x)$ степени $m$, а число $k$ равно кратности корня $\mu$ в уравнении $L(\lambda) = 0$ в резонансном случае и $k = 0$ в нерезонансном.
\end{theorem}

\begin{proof}
	Если $\mu \neq 0$, то заменой $y = ze^{\mu x}$ всегда можно избавиться от $e^{\mu x}$ в правой части. В самом деле, по формуле сдвига после замены имеем, что
	\[L(D)y = L(D)(e^{\mu x}z) = e^{\mu x}L(D + \mu)z = e^{\mu x}P_m(x)\]
	откуда $L(D + \mu)z = P_m(x)$.

	Таким образом, доказательство теоремы осталось провести для уравнения вида
	\[L(D)y = P_m(x)\]

	\begin{enumerate}
		\item Нерезонансный случай: $L(\mu) \neq 0$. Пусть
		      \[P_m(x) = p_mx^m + \cdots + p_0\;\;\; Q_m(x) = q_mx^m + \cdots q_0\]
		      Если подставить и приравнять коэффициенты при одинаковых степенях $x$, получим линейную алгебраическую систему уравнений для определения неизвестных коэффициентов $q_0,\,\cdots,\,q_m$. Матрица системы треугольная с числами $a_n = L(0) \neq 0$, таким образом, все коэффициенты определяются из неё однозначно.
		\item В резонансном случае имеем
		      \[
			      L(\lambda) = \lambda^k(\lambda^{n - k} + a_1\lambda^{n - k - 1} + \cdots + a_{n - k})
		      \]
		      Следовательно,
		      \[
			      L(D) = \begin{cases}
				      D^n + a_1D^{n-1} + \cdots + a_{n - k}D^k,\, k < n \\
				      D^n,\, k = n
			      \end{cases}
		      \]
		      В первом случае замена $D^ky = z$ приводит уравнение к уравнению с нерезонансным случаем. Рассмотрим уравнение
		      \[
			      D^ky = \begin{cases}
				      R_m(x),\, k < n \\
				      P_m(x),\, k = n
			      \end{cases}
		      \]
		      Взяв нулевые начальные условия для этого уравнения
		      \[
			      y(0) = y'(0) = \cdots = y^{(k - 1)}(0) = 0
		      \]
		      получим единственное решение вида
		      \[
			      y(x) = x^kQ_m(x)
		      \]
	\end{enumerate}
\end{proof}


\subsection{Уравнение Эйлера-Коши}
\begin{definition}
	Уравнением Эйлера называется уравнение вида
	\[x^ny^{(n)} + a_1x^{n-1}y^{(n-1)} + \cdots + a_{n-1}xy' + a_ny = 0\]
\end{definition}

\begin{proposition}
	Данное уравнение сводится к уравнению с постоянными коэффициентами при замене $x = e^t$ при $x > 0$ и $x = -e^t$ при $x < 0$.
\end{proposition}

\begin{proof}
	Докажем индукцией по порядку:
	\[\frac{dy}{dx} = \frac{y'_t}{x'_t} = e^{-t}y'_t \;\;\; \frac{d^2y}{dx^2} = e^{-2t}(y''_t - y'_t)\;\;\; \frac{d^ny}{dx^n} = e^{-nt}\phi(y^{(n)}_t,\,y^{(n-1)}_t,\,\cdots,\,y'_t)\]
	Подставим найденные выражения в определение и получим уравнение вида, где $y^{(n)}$ зависит от нового параметра $t$:
	\[b_0y^{(n)} + b_1y^{(n-1)} + b_2y^{(n-2)} + \cdots + b_{n-1}y' + b_n = 0\]
\end{proof}

\subsection{Фундаментальная система решений и общее решение нормальной линейной однородной системы уравнений}
\begin{definition}
	Нормальной системой дифференциальных уравнений называется система дифференциальных уравнений первого порядка, разрешённых относительно производной.

	Нормальная линейная однородная система уравнений выглядит так
	\[\vec{x}(t) = (x_1(t),\,\cdots,\,x_n(t))^T\;\;\; A_{n \times n} = (a_{ij}) \;\;\; \dot{\vec{x}} = A\vec{x}\]
\end{definition}

\begin{theorem}
	Если $\vec{h_1},\,\cdots,\,\vec{h_n}$ -- базис из собственных векторов матрицы $A$, то $\vec{x_i} = e^{\lambda_i t}\vec{h_i}$ -- ФСР для исходной однородной системыю
\end{theorem}

\begin{proof}
	Заметим, что $A(e^{\lambda t}\vec{h}) = e^{\lambda t}(A\vec{h}) = e^{\lambda t}\lambda\vec{h} = (e^{\lambda t}\vec{h})'$, значит собственный вектор является решением. Их линейная независимость следует из того, что, что их вронскиан в точке $t = 0$ равен определителю из координатных столбцов этого базиса, а значит не равен нулю.
\end{proof}

\subsection{Матричная экспонентна, её свойства и применение к решению нормальных линейных систем}
\begin{definition}
	Пусть $t$ -- действительная переменная, $A_{n \times n}$ -- комплекснозначная квадратная матрица. Матричной экспонентой называется ряд:
	\[e^{tA} = E_{n \times n} + \sum_{k = 1}^\infty \frac{t^k}{k!}A^k\]
	Введём обозначение частичных сумм:
	\[S_m = E_{n \times n} + \sum_{k = 1}^m \frac{t^k}{k!}A^k\]
\end{definition}

\begin{definition}
	Матричный ряд
	\[e^{tA} = E_{n \times n} + \sum_{k = 1}^\infty \frac{t^k}{k!}A^k\]
	называется сходящимся при $t_0 \in \mathbb{R}$, если степенной ряд
	\[(S_m)_{ij} = \delta_{ij} + \sum_{k = 1}^\infty \frac{t^k}{k!}a_{ij}^{(k)}\]
	сходится для всех $i,\,j$.
\end{definition}

\begin{lemma}
	$\forall A \in M_{n \times n}(\mathbb{R})$ верно, что ряд $e^{tA} = E + \sum \frac{t^k}{k!}A^k$ сходится абсолютно.
\end{lemma}

\begin{proof}
	Пусть $M = \max\limits_{i,\,j}|a_{ij}|$

	Докажем по индукции: $|a_{ij}^{(k)}| \leq n^{k-1}M^k$
	\begin{enumerate}
		\item База $|a_{ij}^{(1)}| \leq n^0M$
		\item $|a_{ij}^{(k)}| = |\sum_{l = 1}^n a_{il}^{(1)}a_{lj}^{(k-1)}| \leq \sum_{l = 1}^n |a_{il}^{(1)}a_{lj}^{(k-1)}| \leq n\cdot Mn^0\cdot M^{k - 1}n^{k-2} = n^{k-1}M^k$
	\end{enumerate}
	Рассмотрим ряд
	\[1 + \sum_{i = 1}^\infty \frac{|t|^i}{i!}n^{i - 1}M^i\]
	$\lim\limits_{k \to \infty} \frac{a_{k + 1}}{a_k} = \lim\limits_{k \to \infty} \frac{nM}{k + 1} = 0 \Rightarrow$ рассматриваемый ряд сходится по признаку Даламбера и мажорирует каждый компонентный ряд.
\end{proof}

\begin{lemma}
	Формула матричного бинома.

	Если $A$ и $B$ перестановочны, то $\forall n \in \mathbb{N}:\: (A + B)^n = \sum_{i = 0}^n C_n^iA^iB^{n - i}$
\end{lemma}

\begin{lemma}
	Если $A$ и $B$ перестановочны, то $\forall t \in \mathbb{R}$:
	\[e^{tA}e^{tB} = e^{tB}e^{tA} = e^{t(A + B)}\]
\end{lemma}
\begin{proof}
	\begin{align*}
		e^{t(A + B)} = \sum_{n = 0}^\infty \frac{t^n}{n!}(A + B)^n = \sum_{n = 0}^\infty \sum_{k + m = n} \frac{t^kA^k}{k!}\frac{t^mB^m}{m!} \overset{\text{абс.сход.}}{=} \sum_{k = 0}^\infty\sum_{m=0}^\infty \frac{t^kA^k}{k!}\frac{t^mB^m}{m!} = \\ \sum_{k = 0}^\infty\frac{t^kA^k}{k!}\sum_{m=0}^\infty\frac{t^mB^m}{m!} = e^{tB}e^{tA} = e^{tA}e^{tB}
	\end{align*}
\end{proof}

\begin{lemma}
	Свойства матричной экспонетны.

	\begin{enumerate}
		\item Если $S$ -- невырожденная и $A = SBS^{-1}$, то $e^{tA} = Se^{tB}S^{-1},\, \forall t \in \mathbb{R}$
		\item $(e^{tA})'_t = Ae^{tA} = e^{tA}A$
	\end{enumerate}
\end{lemma}
\begin{proof}
	\begin{enumerate}
		\item Заметим, что $A^k = SB^kS^{-1}$:
		      \[e^{tA} = \sum_{k=0}^\infty\frac{t^k}{k!}A^k = S \left(\sum_{k = 0}^\infty \frac{t^k}{k!}B^k\right)S^{-1} = Se^{tB}S^{-1}\]
		\item
		      \[\frac{d}{dt}e^{tA} = \frac{d}{dt}\left(\sum_{k=0}^\infty\frac{t^k}{k!}A^k\right) = \sum_{k = 1}^\infty \frac{t^{k-1}}{(k-1)!}A^k = A \sum_{k=0}^\infty\frac{t^k}{k!}A^k = Ae^{tA}\]
	\end{enumerate}
\end{proof}

\begin{theorem}
	Матричная экспонентна для ФСР.

	Матрица $e^{tA}$ является фундаментальной матрицей для системы линейных уравнений $\dot{\vec{x}} = A\vec{x}$
\end{theorem}

\begin{proof}
	$(e^{tA})' = Ae^{tA}$, следовательно, каждый столбец матрицы $e^{tA}$ является решением исходной системы. Поскольку $\det e^{tA} \neq 0,\, \forall t$, то $e^{tA}$ фундаментальна.
\end{proof}

\begin{note}
	Общее решение системы $\dot{\vec{x}} = A\vec{x}$ -- это $e^{tA}\vec{c}$, где $\vec{c}$ -- вектор констант.
\end{note}

\begin{theorem}
	Общее решение системы $\dot{\vec{x}} = A\vec{x} + \vec{f}(t)$ задаётся следующей формулой:
	\[\vec{x} = e^{tA}\left(\int_{t_0}^te^{-\tau A}\vec{f}(\tau)d\tau + \vec{c_0}\right)\]
\end{theorem}
\begin{proof}
	Метод вариации постоянных:
	\begin{align*}
		\vec{x} = e^{tA}\vec{c}(t) \Rightarrow                                                                        \\
		(e^{tA}\vec{c}(t))' = Ae^{tA}\vec{c}(t) + e^{tA}\dot{\vec{c}}(t) = Ae^{tA}\vec{c}(t) + \vec{f}(t) \Rightarrow \\
		e^{tA}\dot{\vec{c}}(t) = \vec{f}(t) \Rightarrow                                                               \\
		\dot{\vec{c}}(t) = e^{-tA}\vec{f}(t) \Rightarrow                                                              \\
		\vec{c}(t) = \int_{t_0}^t e^{-\tau A}\vec{f}(\tau)d\tau + \vec{c_0}
	\end{align*}
\end{proof}

\section{Линейные дифференциальные уравнения и линейные системы дифференциальных уравнений с переменными коэффициентами}
\begin{note}
	Линейным дифференциальным уравнением с переменными коэффициентами порядка $n$ называется уравнение
	\[y^{(n)} + a_1(x)y^{(n-1)}+\cdots + a_n(x)y = f(x)\]
	Решение этого уравнение всегда можно свести к решению системы линейных уравнений порядка $n$ аналогично случаю с постоянными коэффициентами.
\end{note}

\subsection{Теоремы существования и единственности решения задачи Коши для нормальной линейной системы уравнений и для линейного уравнения n-го порядка}
\begin{theorem}
	Существования и единственности для системы.

	Зададим начальное условие $y(x_0) = y_0$, где $x \in I$ и $y_0$ -- заданный $n$-мерный вектор. Пусть матрица $A(x)$ и $\vec{f}(x)$ непрерывны на $I$. Тогда на всём $I$ решение задачи Коши существует и единственно.
\end{theorem}

\begin{proof}
	Эта система является частным случаем уже рассмотренной задачи Коши в нормальном виде с
	\[f_i(x,\, y_1,\,\cdots,\,y_n) = a_{i1}(x)y_1 + a_{i2}(x)y_2+\cdots+a_{in}(x)y_n + \vec{f_i}(x),\;\;\; i = \overline{1,\,n}\]
	Эти функции $f_i$ определены и непрерывны при $x \in I,\, (y_1,\,\cdots,\,y_n)\in \mathbb{R}^n$ и удовлетворяют условию Липшица с постоянной
	\[L = \max_{1\leq i,\,j \leq n}\max_{x \in I}|a_{ij}(x)|\]
	Следовательно, требуемые условия выполнены и она имеет единственное решение на $I$.
\end{proof}

\begin{theorem}
	Существования и единственности для уравнения.

	Пусть все функции $a_i(x),\, i = \overline{1,\,n}$ и $f(x)$ -- непрерывны на $I$ и пусть $x_0 \in I$.

	Тогда при произвольных начальных значениях $y_1^{(0)},\, \cdots,\, y_n^{(0)}$ решение задачи Коши
	\[y(x_0) = y_1^0,\, y'(x_0) = y_2^0,\,\cdots,\,y^{(n-1)}(x_0) = y_n^0\]
	существует и единственно на всём $I$.
\end{theorem}

\begin{proof}
	Данное уравнение также является частным случаем уже рассмотренного общего случая с функцией
	\[F(x,\,y,\,y',\,\cdots,\,y^{(n-1)}) = f(x) - a_n(x)y - a_{n-1}(x)y' - \cdots - a_1(x)y^{(n-1)}\]
	Эта функция $F$ определена и непрерывна при $x \in I,\, (y,\, y',\,\cdots,\,y^{(n-1)})\in\mathbb{R}^n$ и удовлетворяет условию Липшица с постоянной
	\[L = \max_{1\leq i,\,j\leq n}\max_{x \in I} |a_i(x)|\]
	Следовательно, для задачи Коши выполнены условия теоремы и её решение существует и единственно на $I$.
\end{proof}

\subsection{Фундаментальная система и фундаментальная матрица решений линейной однородной системы}
\begin{note}
	Теперь мы рассматриваем однородную систему
	\[\vec{y'} = A(x)\vec{y}(x)\]
	где $x \in I$ и $A(x)$ -- непрерывны на $I$. Комплекснозначная матрица $A(x)$ порядка $n$. Решением будет являться комплекснозначная вектор-функция $y$.
\end{note}

\begin{lemma}
	Принцип суперпозиции.

	Если $y_1(x),\, y_2(x)$ -- решение данной системы, то линейная комбинация $y = c_1y_1(x) + c_2y_2(x)$ также решение.
\end{lemma}

\begin{definition}
	Вектор-функции $y_1,\,\cdots,\,y_k$ называются линейно зависимыми на промежутке $I$, если найдутся числа $c_1,\,\cdots,\,c_k$ одновременно неравные нулю, что
	\[c_1y_1(x) + \cdots + c_ky_k(x) \equiv 0,\, \forall x \in I\]
	В противном случае функции называются линейно независимыми
\end{definition}

\begin{lemma}
	Если $y_1,\,\cdots,\,y_k$ линейно зависимы на промежутке $I$, то числовые вектора
	\[y_1(x),\,\cdots,\,y_k(x),\, \forall x \in I\]
	линейно зависимы. Обратное утверждение неверно.
\end{lemma}

\begin{theorem}
	Пусть $y_1(x),\,\cdots,\,y_k(x)$ -- решения данной однородной системы. Эти решения линейно независимы на $I$ тогда и только тогда, когда
	\[\forall x_0 \in I:\: y_1(x_0),\,\cdots,\,y_k(x_0)\]
	линейно независимы как числовые векторы.
\end{theorem}

\begin{proof}
	$\Rightarrow$ Пусть $y_1(x),\,\cdots,\,y_k(x)$ -- линейно независимые решения линейной однородной системы. Если существует $x_0 \in I$, что $y_1(x_0),\,\cdots,\,y_k(x_0)$ линейно независимы, то найдётся нетривиальная линейная комбинация, что $y = \sum c_iy_i(x_0) = 0$

	Вектор-функция $y$ является решением данной системы по принципу суперпозиции и также удовлетворяет начальному условию $y(x_0) = 0$. Но тогда $y(x) \equiv 0$ на $I$ по теореме о существовании и единственности. Но тогда $y_1(x),\,\cdots,\,y_k(x)$ линейно зависимы. Противоречие.

	$\Leftarrow$ Пусть $\forall x_0 \in I:\: y_1(x_0),\,\cdots,\,y_k(x_0)$ линейно независимы на $I$. Пусть $y_1(x),\,\cdots,\,y_k(x)$ линейно зависимы, тогда по предыдущей лемме $y_1(x_0),\,\cdots,\,y_k(x_0)$ также линейно зависимы. Противоречие.
\end{proof}

\begin{theorem}
	Для данной системы фундаментальная система существует и их бесконечное число.
\end{theorem}

\begin{proof}
	Фиксируем $x_0 \in I$ и $n$ линейно независимых числовых векторов $y_1^0,\,\cdots,\,y_n^0$ с $n$ компонентами. Обозначим через $\phi_j(x)$ решение данной системы, удовлетворяющее начальному условию $y_j(x_0) = y_j^0$. По теореме о существовании и единственности каждое такое решение существует и единственно на $I$. Но тогда система решений $\phi_j(x)$ образует фундаментальную систему решений, так как она была построена из линейно независимых числовых векторов. Точку $x_0 \in I$ и векторы можно выбирать бесконечным числом способов.
\end{proof}

\begin{theorem}
	Если $\phi_1^0,\,\cdots,\,\phi_n^0$ -- фундаментальная система решений данной системы, то любое решение представимо в виде линейной комбинации членов фундаментальной системы.
\end{theorem}

\begin{proof}
	Пусть $x_0 \in I,\, y$ -- некоторое решение системы. Тогда $\phi_1^0(x_0),\,\cdots,\,\phi_n^0(x_0)$ линейно независимы по определению и числовой вектор единственным образом выражается через линейную комбинацию этих векторов. В силу единственности задачи Коши, коэффициенты линейной комбинации окажутся одними и теми же для всех точек отрезка.
\end{proof}

\begin{definition}
	Матрица $\Phi(x)$, у которой столбцы образуют фундаментальную систему решений, называется фундаментальной матрицей данной системы.
\end{definition}

\begin{lemma}
	Если $Y_1(x),\, Y_2(x)$ -- фундаментальные матрицы одной системы, то существует невырожденная числовая матрица $C$, что $Y_1 \equiv Y_2C$
\end{lemma}

\subsection{Структура общего решения линейной однородной и неоднородной систем}
\begin{theorem}
	О структуре решения неоднородной системы.

	Пусть $y_0(x)$ -- некоторое частное решение неоднородной системы и $\Phi(x)$ -- фундаментальная матрица соответствующей однородной системы.

	Тогда все решения исходной системы задаются формулой
	\[\vec{y}(x) = \vec{y_0} + \Phi(x)\vec{C}\]
	где $\vec{C}$ -- произвольный числовой вектор размерности $n$.
\end{theorem}

\begin{proof}
	В исходной системе сделаем замену
	\[\vec{y}(x) = \vec{z}(x) + \vec{y_0}(x)\]
	Тогда получим, что $\vec{z}(x)$ удовлетворяет соответствующей однородной системе. Общее решение однородной системы записывается, как
	\[\vec{z}(x) = \Phi(x)\vec{C}\]
	Из замены следует утверждение теоремы.
\end{proof}

\subsection{Определитель Вронского. Формула Лиувилля-Остроградского.}
\begin{theorem}
	Лиувилля-Остроградского.

	Пусть $W(x)$ -- вронскиан решений $y_1(x),\,\cdots,\,y_n(x)$ системы $y'(x) = A(x)y(x)$ на промежутке $I$ и $x_0 \in I$, тогда $\forall x \in I$ имеет место формула Лиувилля-Остроградского:
	\[W(x) = W(x_0) \cdot \exp\left(\int_{x_0}^x \text{tr}A(t)dt\right)\]
\end{theorem}

\begin{proof}
	Докажем, что $W(x)$ удовлетворяет дифференциальному уравнению
	\[W'(x) = \text{tr}A(x)\cdot W(x)\;\;\; x \in I\]
	Пусть $y_{ij}(x),\, i=\overline{1,\,n}$ компоненты решения $y_j(x),\, j=\overline{1,\,n}$. Тогда $W(x)$ является функцией всех этих компонент:
	\[W(x) = W[y_{11}(x),\,y_{21}(x),\,\cdots,\,y_{nn}(x)]\]
	По формуле производной сложной функции получаем, что
	\[W'(x) = \sum_{p,\,q = 1}^n \frac{\partial W}{\partial y_{pq}}(x)y'_{pq}(x)\]
	Если $W_{pr}(x)$ -- алгебраическое дополнение $y_{pr}(x)$ в $W(x)$, то разложение $W(x)$ по $p$-й строке даёт
	\[W(x) = \sum_{r = 1}^n y_{pr}(x)W_{pr}(x)\]
	Отсюда находим, что
	\[\frac{\partial W}{\partial y_{pq}}(x) = W_{pq}(x)\]
	Каждая вектор-функция удовлетворяет системе $y'(x) = A(x)y(x)$, то есть
	\[y'_q(x) = A(x)y'_q(x),\;\;\; q=\overline{1,\,n},\, x \in I\]
	Отсюда находим, что
	\[y'_{pq}(x) = \sum_{r = 1}^n a_{pr}(x)y_{rq}(x)\]
	Подставляя найденные выражения в формулу $W'(x)$ получим, что
	\[W'(x) = \sum_{p,\,q = 1}^n W_{pq}(x)\sum_{r = 1}^n a_{pr}(x)y_{rq}(x) = \sum_{p,\,r = 1}^n a_{pr}(x) \sum_{q = 1}^n y_{rq}(x)W_{pq}(x)\]
	Но по свойствам определителя мы знаем, что
	$\sum_{q = 1}^n y_{rq}(x)W_{pq}(x) = \delta_{pr}W(x) \Rightarrow$
	\[W'(x) = W(x)\sum_{p,\,r = 1}^n a_{pr}(x) \delta_{pr} = W(x) \sum_{p = 1}^n a_{pp}(x) = W(x)\text{tr}A(x)\]
	Интегрирование этого линейного однородного уравнения первого порядка даёт искомую формулу.
\end{proof}

\begin{note}
	Формула Лиувилля-Остроградского для однородного уравнения $n$-го порядка:
	\[y^{(n)} + a_1(x)y^{(n-1)}+\cdots + a_n(x)y = 0\]
	\[\begin{cases}
			y_1' = y_2     \\
			y_2' = y_3     \\
			\cdots         \\
			y_{n-1}' = y_n \\
			y_n' = -a_ny_1 - a_{n-1}y_2 - \cdots - a_1y_n
		\end{cases} \Rightarrow \text{tr}A(x) = -a_1(x) \Rightarrow W(x) = W(x_0)\exp\left(-\int_{x_0}^x a_1(t)dt\right)\]
\end{note}

\subsection{Метод вариации постоянных для линейной неоднородной системы уравнений и для линейного неоднородного уравнения n-го порядка}
Как мы уже поняли, все уравнения $n$-го порядка сводятся к системам, поэтому достаточно показать для системы.

\begin{enumerate}
	\item Найти ФСР однородной системы (и фундаментальную матрицу $Y$).
	\item Продифференцировать вектор-функцию $\vec{y}(x) = Y(x)\vec{C}(x)$, где $\vec{C}(x)$ -- вектор-функция:
	      \[\vec{y'} = Y'\vec{C} + Y\vec{C'} = AYC + f\]
	\item Выразить и проинтегрировать $\vec{C'}$, после чего выразить частное решение неоднородной системы:
	      \[Y\vec{C'} = f \Rightarrow \vec{C'} = Y^{-1}f \Rightarrow \vec{C} = \int_{x_0}^xY^{-1}(t)\vec{f}(t)dt + \vec{C_0}\]
	      \[\vec{y}(x) = Y(x)\int_{x_0}^xY^{-1}(t)\vec{f}(t)dt + Y(x)\vec{C_0}\]
\end{enumerate}

\subsection{Краевая задача, теорема об альтернативе}
\begin{definition}
	Задача нахождения функции $y = y(x),\, y \in C^2[a,\,b]$ из условий
	\[
		\begin{cases}
			y'' + a_1(x)y' + a_2(x)y = f(x)                                 \\
			\gamma_1y'(a) + \gamma_2y(a) + \gamma_3y'(b) + \gamma_4y(b) = 0 \\
			\delta_1y'(a) + \delta_2y(a) + \delta_3y'(b) + \delta_4y(b) = 0
		\end{cases}
	\]
	где $\gamma_{1,\,2,\,3,\,4},\, \delta_{1,\,2,\,3,\,4}$ -- некоторые числа такие, что
	\[\exists i,\, j:\: \gamma_i \cdot \delta_j \neq 0\]
\end{definition}

\begin{theorem}
	Об альтернативе.

	Для линейного неоднородного уравнения $n$-го порядка с $n$ линейными краевыми условиями возможны только два случая:
	\begin{enumerate}
		\item Задача имеет единственное решение при любых правых частях в уравнении и краевых условиях.
		\item Однородная задача имеет бесконечно много решений, а неоднородная задача при некоторых правых частях имеет бесконечно много решений, а при всех других -- не имеет решений.
	\end{enumerate}
\end{theorem}

\begin{proof}
	Общее решение данного уравнения имеет вид
	\[y(x) = \sum_{i = 1}^n c_iy_i(x) + y_0(x)\]
	где $y_1,\,\cdots,\,y_n$ -- линейно независимые решения однородного уравнения, $y_0(x)$ -- частное решение. Подставляя это общее решениев в краевые условия и перенося $y_0(x)$ в правую часть, получаем систему $n$ линейных алгебраических уравнений относительно $c_1,\,\cdots,\,c_n$. Коэффициенты системы зависят только от значений $y,\, y',\,\cdots$ в заданных точках и не зависят от правых частей уравнения и краевых условий. Если данная задача однородна, то правые части алгебраических уравнений равны нулю.

	Возможны только два следующих случая:
	\begin{enumerate}
		\item Если детерминант не равен нулю, то система имеет единственное решение $c_1,\,\cdots,\,c_n$ при любых правых частях. Подставляя эти значения в общее решение, получаем единственное решение краевой задачи.
		\item Если детерминант системы равен нулю, то однородная система имеет бесконечно много решений относительно $c_1,\,\cdots,\,c_n$, а неоднородная система имеет решение не при любых правых частях. Если она имеет решение, то она имеет бесконечно много решений, так как к этому решению можно прибавить любое решение однородной системы, умноженное на любую константу.
	\end{enumerate}
\end{proof}

\subsection{Теорема Штурма и следствия из неё}
Будем рассматривать уравнение второго порядка:
\[y'' + a(x)y' + b(x)y = 0\]
где $a(x),\,b(x)$ -- действительно-значные функции, заданные на $I \subseteq \mathbb{R},\, a(x),\, b(x) \in C^1(I)$.

\begin{definition}
	Решение называется нетривиальным на $I$, если оно не равно нулю хотя бы в одной точке $x_0 \in I$.
\end{definition}

\begin{definition}
	Решение, имеющее более одного нуля на $I$, называется колеблющимся.
\end{definition}

\begin{definition}
	Значение $x_0$ называется простым нулём функции $f(x)$, определённой и дифференцируемой в точке $x_0$, если $f(x_0) = 0$ и $f'(x_0) \neq 0$. Если производная равна $0$, то такое значение называется кратным нулём.
\end{definition}

Выполним в изначальном уравнении замену $y(x) = u(x) \cdot z(x)$, так как хотим понизить порядок. Приэтом $z(x)$ -- неизвестная функция, а $u(x)$ подберём так, как нам нужно.
\[y' = u'z + uz' \;\;\;\; y'' = u''z + 2u'z' + uz''\]
Подставляем в изначальное уравнение:
\[uz'' + (2u' + au)z' + (u'' + au' + bu)z = 0\]
Возьмём $u(x) = \exp\left[-\frac{1}{2}\int_{x_0}^xa(t)dt\right]$, чтобы коэффициент перед $z'$ сократился. Заметим, что $u(x)$ в ноль никогда не обращается. Подставим $u(x)$ в полученное выше уравнение:
\[z''(x) + q(x)z = 0\;\;\;\;\; q(x) = \frac{u'' + au' + bu}{u}\]
Полученное уравнение называется приведённым. (К такому уравнению можно привести всегда.)

\begin{lemma}
	Если $z(x)$ -- нетривиальное решение приведённого уравнения, то каждый его нуль на промежутке $I$ является простым.
\end{lemma}

\begin{proof}
	Пусть это не так, то есть $x_0 \in I$ -- кратный нуль:
	\[
		\begin{cases}
			z''(x) + q(x)z = 0\\
			z(x_0) = 0\\
			z'(x_0) = 0
		\end{cases}
	\]
	решение полученной ЗК единственно. Более того, оно равно нулю на всём $I$ -- противоречие.
\end{proof}

\begin{lemma}
	Нули любого нетривиального решениея приведённого уравнения не имеют конечной предельной точки на $I$.
\end{lemma}

\begin{proof}
	Пусть это не так, то есть существует $\{x_n\}$ -- последовательность нулей $z(x)$, которая сходится к $x_0 \in I$. В силу непрерывности $z(x)$ имеем $\lim_{x \in x_0}z(x) = z(x_0) = \lim_{n \to \infty} z(x_n) = 0$. Значит $x_0$ -- тоже нуль функции $z(x)$. Далее, по определению
	\[z'(x_0) = \lim_{n \to \infty} \frac{z(x_n) - z(x_0)}{x_n - x_0} = 0\]
	таким образом, $x_0$ оказался кратным нулём, что противоречит предыдущей лемме.
\end{proof}

\begin{corollary}
	Любое нетривиальное на $[\alpha,\, \beta] \subseteq I$ решение данного уравнения имеет не более конечного числа нулей на этом отрезке. 
\end{corollary}

\begin{proof}
	От противного. Если их счётное количество, то для них получаем предельную точку нулей по теореме Больцано-Вейшерштрасса, отсюда противоречие предыдущей лемме.
\end{proof}

\begin{theorem}
	Теорема Штурма.

	Рассмотрим $y'' + q(x)y = 0$ и $y'' + Q(x)y = 0$. Пусть $\forall x \in I:\: q(x) \leq Q(x)$ и пусть $y(x),\,z(x)$ -- нетривиальные решения соответствующих уравнений. Если $x_1 \leq x_2$ -- последовательные нули $y(x)$, то
	\begin{itemize}
		\item Либо $z(x_1) = z(x_2) = 0$
		\item Либо существует хотя бы одна точка $x_0 \in (x_1,\,x_2)$, для которого выполнено $z(x_0) = 0$
	\end{itemize}
\end{theorem}

\begin{proof}
	Будем действовать от противного: предположим, что $z(x) \neq 0$ на $(x_1,\,x_2)$, то есть, например, $z(x) > 0$. Без потери общности предположим, что $y(x) > 0$ на $(x_1,\,x_2)$. Из условия теоремы: $y(x_1) = y(x_2) = 0$. Тогда
	\[y'(x_1) = \lim_{x \to x_1 + 0} \frac{y(x) - y(x_1)}{x - x_1} \geq 0\;\;\;\;\; y'(x_2) = \lim_{x \to x_2 - 0} \frac{y(x) - y(x_2)}{x - x_2} \leq 0\]
	Но поскольку $x_1,\,x_2$ -- простые нули, то $y'(x_1) > 0,\, y'(x_2) < 0$. Посчитаем $(y'' + q(x)y)z - (z'' + Q(x)z)y = 0$:
	\[y''z - z''y = (Q - q)yz\]
	Вычислим интеграл левой части:
	\begin{align*}
		\int_{x_1}^{x_2}(y''z - z''y)dx = \int_{x_1}^{x_2}zdy' - \int_{x_1}^{x_2}ydz' = \\
		(zy')|_{x_1}^{x_2} - \int_{x_1}^{x_2}y'dz - (yz')|_{x_1}^{x_2} + \int_{x_1}^{x_2}z'dy =
	\end{align*}
	При этом верно: $dz = z'dx,\, dy = y'dx$. Поэтому полученные интегралы совпадают и сокращаются.
	\[= (zy')|_{x_1}^{x_2} - (yz')|_{x_1}^{x_2} = (zy')|_{x_1}^{x_2}\]
	Тогда исходная разность принимает вид:
	\[z(x_2)y'(x_2) - z(x_1)y'(x_1) = \int_{x_1}^{x_2}(Q - q)yzdx \geq 0\]
	Далее возможны следующие варианты:
	\begin{itemize}
		\item $z > 0$ на $[x_1,\,x_2]$
		\item $z > 0$ на $[x_1,\,x_2)$, и $z(x_2) = 0$
		\item $z > 0$ на $(x_1,\,x_2]$, и $z(x_1) = 0$
	\end{itemize}
	Заметим, что левая часть полученного выражения во всех случаях отрицательная $\Rightarrow$ противоречие.
\end{proof}

\begin{corollary}
	Если $q(x) \leq 0$, то любое нетривиальное решение уравнения $y'' + q(x)y = 0$ имеет не более одного нуля.
\end{corollary}

\begin{proof}
	Предположим, что есть нетривиальное решение $y(x)$, у которого есть 2 последовательных нуля.

	Тогда пусть $Q(x) \equiv 0$, то есть $z'' = 0 \Rightarrow z = ax + b \neq 0$.

	Возьмём $z \equiv 42$. Но эта функция не имеет ни одного нуля $\Rightarrow$ противоречие с теоремой Штурма.
\end{proof}

\begin{corollary}
	Пусть $y_1(x),\,y_2(x)$ -- линейно независимые решения уравнения $y'' + q(x)y=0$. Если $x_1,\, x_2$ -- последовательные нули $y_1$, то между ними имеется ровно один нуль $y_2$.
\end{corollary}

\begin{proof}
	Положим $q(x) \equiv Q(x)$. Заметим, что $y_1,\,y_2$ не обращаются одновременно в нуль в $x_1,\, x_2$ (ввиду линейной независимости). Если предположить, что между $x_1,\,x_2$ лежит хотя бы два нуля $y_2$, то аналогичным образом получается, что между этими нулями есть хотя бы ещё один корень $y_1$.
\end{proof}

\begin{corollary}
	Если некоторое нетривиальное решение уравнения $y'' + q(x)y = 0$ имеет бесконечно много нулей, то и любое другое решение также имеет бесконечно много нулей.
\end{corollary}

\end{document}