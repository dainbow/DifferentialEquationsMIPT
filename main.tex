\documentclass[a4paper,12pt]{article}

%%% Работа с русским языком

\usepackage{cmap}					% поиск в PDF
\usepackage{mathtext} 				% русские буквы в формулах
\usepackage[T2A]{fontenc}			% кодировка
\usepackage[utf8]{inputenc}			% кодировка исходного текста
\usepackage[english,russian]{babel}	% локализация и переносы
\usepackage{indentfirst}            % красная строка в первом абзаце
\usepackage[unicode]{hyperref}
\usepackage{epigraph}
\frenchspacing                      % равные пробелы между словами и предложениями

%%% Дополнительная работа с математикой
\usepackage{amsmath,amsfonts,amssymb,amsthm,mathtools} % пакеты AMS
\usepackage{bbm} % Blackboard bold для цифр
\usepackage{icomma}                                    % "Умная" запятая

\renewcommand{\phi}{\ensuremath{\varphi}}
\renewcommand{\kappa}{\ensuremath{\varkappa}}
\renewcommand{\le}{\ensuremath{\leqslant}}
\renewcommand{\leq}{\ensuremath{\leqslant}}
\renewcommand{\ge}{\ensuremath{\geqslant}}
\renewcommand{\geq}{\ensuremath{\geqslant}}
\renewcommand{\emptyset}{\ensuremath{\varnothing}}

\newcommand{\cl}{\text{cl }}
\newcommand{\setint}{\text{int }}

\theoremstyle{plain}
\newtheorem{theorem}{Теорема}[section]
\newtheorem{lemma}{Лемма}[section]
\newtheorem{proposition}{Утверждение}[section]
\newtheorem*{corollary}{Следствие}
\newtheorem*{exercise}{Упражнение}

\theoremstyle{definition}
\newtheorem{definition}{Определение}[section]
\newtheorem*{note}{Замечание}
\newtheorem*{reminder}{Напоминание}
\newtheorem*{example}{Пример}
\newtheorem*{tasks}{Вопросы и задачи}

\theoremstyle{remark}
\newtheorem*{solution}{Решение}

%%% Оформление страницы
\usepackage{extsizes}     % Возможность сделать 14-й шрифт
\usepackage{geometry}     % Простой способ задавать поля
\usepackage{setspace}     % Интерлиньяж
\usepackage{enumitem}     % Настройка окружений itemize и enumerate
\setlist{leftmargin=25pt} % Отступы в itemize и enumerate

\geometry{top=25mm}    % Поля сверху страницы
\geometry{bottom=30mm} % Поля снизу страницы
\geometry{left=20mm}   % Поля слева страницы
\geometry{right=20mm}  % Поля справа страницы

\begin{document}
\tableofcontents
\newpage

\section{Основные понятия, простейшие типы дифференциальных уравнений}
\subsection{Основные понятия}
\begin{definition}
	Уравнение вида
	\begin{equation}
		\label{SDE}
		F(x,\, y,\, \ldots,\, y^{(n)}) = 0
	\end{equation}

	называется обыкновенным дифференциальным уравнением $n$-го порядка.
\end{definition}

\begin{definition}
	Функция $\phi(x)$, определённая на $I$ вместе со своими $n$ производными, называется решением уравнения (\ref*{SDE}), если:

	\begin{enumerate}
		\item $\phi$ и все её $n$ производных непрерывны на $I$.
		\item $\forall x \in I:\: (x,\, \phi(x),\, \phi'(x),\, \ldots,\, \phi^{(n)}(x)) \in \Omega$, где $\Omega$ - область определения $F$.
		\item $\forall x \in I:\: F(x,\, \phi(x),\, \phi'(x),\,\ldots,\,\phi^{(n)}) = 0$
	\end{enumerate}
\end{definition}

\begin{definition}
	Решение $y = \phi(x),\, x \in \langle a,\, b\rangle$ уравнения (\ref*{SDE}) называется продолжаемым вправо, если существует такое решение $y = \psi(x),\, x \in \langle a,\, b_1 \rangle,\, \langle a,\, b\rangle \subset \langle a,\, b_1 \rangle$, что $\phi(x) \equiv \psi(x)$ при $x \in \langle a,\, b\rangle$

	Аналогично определяется продолжение решения влево.
\end{definition}

\begin{definition}
	Решение называется непродолжаемым, если его нельзя продолжить ни вправо, ни влево.
\end{definition}

\begin{definition}
	Система дифференциальных уравнений называется автономной, если она имеет вид:
	\[
		\frac{dx_k}{dt} = f_k(x_1,\, \ldots,\,x_n);\; k = 1,\,\ldots,\,n
	\]
	также очень часто автономные системы записываются в компактном векторном виде:
	\begin{equation}
		\label{Autonomius}
		\dot{x} = F(x),\, x \in \Omega \subseteq E^n
	\end{equation}

\end{definition}

\begin{definition}
	Непрерывно дифференцируемая в $\Omega$ функция $u(x)$ называется первым интегралом системы $(\ref*{Autonomius})$, если $\forall t \in T:\: u(x(t)) \equiv const$ для каждого решения $x(t)$ этой системы.
\end{definition}

\begin{definition}
	Если поставить в соответствие каждой точке $(x,\, y)$ некоторого множества $\Omega \subseteq E^2$ вектор с координатными представлением $(1,\, f(x,\,y))$, то полученное векторное множество принято называть полем направлений ОДУ первого порядка.
\end{definition}

\begin{definition}
	Векторное поле - это отображение, которое сопоставляет каждой точке некоторого пространства вектор
\end{definition}

\begin{definition}
	Пусть $x(t)$ есть частное решение системы (\ref*{Autonomius}), тогда вектор-функция $x(t),\, t \in T$, параметрически задаёт некоторую линию в $E^n$, называемую фазовой траекторией этой системы.
\end{definition}

\begin{definition}
	Совокупность фазовых траекторий для всех частных решений будем именовать фазовым портретом системы (\ref*{Autonomius})
\end{definition}

\begin{definition}
	График функции $y = \phi(x)$ можно рассматривать как геометрическое представление частного решения уравнения (\ref*{SDE}). Этот график обычно называют интегральной кривой уравнения (\ref*{SDE}).
\end{definition}

\subsection{Простейшие типы уравнений первого порядка}

\subsubsection*{Уравнения с разделяющимися переменными}
\begin{definition}
	Уравнения с разделяющимися переменными - это уравнения, которые могут быть записаны в виде

	\[y' = f(x)g(y)\; f(x) \in C(I_1),\, g(y) \in C(I_2)\]

	или же в виде

	\[M(x)N(y)dx + P(x)Q(y)dy = 0\]
\end{definition}

\begin{note}
	Если же $y_k \in I_2$ решение уравнения $g(y) = 0$, то $y \equiv y_k$ -- решение дифф. уравнения
\end{note}

Если же $y(x)$ нигде не принимает значение $y_k$, то $g(y) \neq 0$, а потому мы можем делить на него. Значит, чтобы решить исходное уравнение, необходимо разделить переменные, то есть, привести уравнение к такой форме, чтобы при дифференциале $dx$ стояла функция, зависящая лишь от $x$, а при дифференциале $dy$ -- функция, зависящая от $y$.

\subsubsection*{Однородные уравнения}
\begin{definition}
	Функция двух переменных $f(x,\,y)$ называется однородной степени $m$, если для всех $t$ справедливо соотношение:
	\[f(tx,\,ty) = t^mf(x,\,y)\]
\end{definition}

\begin{definition}
	Однородным дифференциальным уравнением называется уравнение вида
	\[M(x,\,y)dx + N(x,\,y)dy = 0\]
	если $M(x,\,y)$ и $N(x,\,y)$ -- однородные функции одной и той же степени $m$.
\end{definition}

Однородное уравнение приводится к уравнению с разделяющимися переменными с помощью замены искомой функции $y(x)$ по формуле:
\[t(x) = \frac{y(x)}{x}\]

Тогда производная $y'$ и дифференциал $dy$ заменяются по формулам:
\[y' = t'x + t,\,\, dy = tdx + xdt\]

\subsubsection*{Линейные уравнения}
\begin{definition}
	Линейным уравнением первого порядка называется уравнение, линейное относительно искомой функции $y(x)$ и её производной, то есть, уравнения вида
	\[y' + a(x)y = b(x)\;\;\; a(x),\,b(x) \in C(I)\]

	Функция $b(x)$ называется свободным членом уравнения.

	Уравнение \[y' + a(x)y = 0\] называется линейным однородным уравнением, соответствующим изначальному линейному уравнению.
\end{definition}

Покажем, что однородное уравнение является уравнением с разделяющимися перменными

\[y' + a(x)y = 0 \Rightarrow \int \frac{dy}{y} = -\int a(x) dx \Rightarrow |y| = e^C \cdot e^{-\int_{x_0}^x a(t) dt}\]

Объединяя все решения, получаем общее решение:
\[y_0 = C \exp\left[-\int_{x_0}^x a(t) dt\right]\]

Будем искать частное решение исходного линейного уравнения методом вариации постоянной:
\[y_{\textbf{ч}} = C(x) \cdot \exp\left[-\int_{x_0}^x a(t) dt\right]\]

\subsubsection*{Уравнение Бернулли}
\begin{definition}
	Нелинейное уравнение первого порядка вида
	\[y' + a(x)y = b(x)y^m,\;\;\; m \neq 0,\, m \neq 1,\, a,\,b \in C(I)\]
	называется уравнением Бернулли.
\end{definition}

Заметим, что $y = 0$ -- решение уравнения Бернулли при $m > 0$.

Если $y \neq 0$, то, разделив уравнение на $y^m$ и вводя новую неизвестную функцию $z = y^{1 - m}$, относительно функции $z$ получаем линейное уравнение.

\subsubsection*{Уравнение Рикатти}
\begin{definition}
	Нелинейное уравнение первого порядка вида
	\[y' = a(x)y^2 + b(x)y + c(x) \;\;\; a,\,b\,\,c \in C(I)\]
	называется уравнением Рикатти
\end{definition}

В отличие от всех уравнений, рассматривавшихся ранее, уравнение Рикатти не всегда интегрируется в квадратурах. Чтобы решить его, необходимо знать хотя бы одно частное решение $y = y_1(x)$ этого уравнения. Тогда замена $y = y_1 + z$ приводит это уравнение к уравнению Бернулли.

\subsubsection*{Логистическое уравнение Ферхюльста}
\begin{note}
	Исходные предположения для вывода уравнения при рассмотрении популяционной динамики выглядит следующим образом:
	\begin{itemize}
		\item Скорость размножения популяции пропорциональна её текущей численности при прочих равных условиях
		\item Скорость размножения популяции пропорциональна количеству доступных ресурсов при прочих равных условиях.
	\end{itemize}
\end{note}

\begin{definition}
	Обозначая через $P$ численность популяции, а время - $t$, модель можно свести к дифференциальному уравнению
	\[\frac{dP}{dt} = rP(1 - \frac{P}{K})\]
	где параметр $r$ характеризует скорость роста, а $K$ -- максимальную возможную численность популяции.
\end{definition}

\begin{note}
	Точным решения является логистическая функция, S-образная кривая:
	\[P(t) = \frac{KP_0e^{rt}}{K + P_0(e^{rt}-1)}\] где $P_0$ -- начальная популяция, и $\lim\limits_{t \to \infty} P(t) = K$.
\end{note}

\subsection{Уравнения в полных дифференциалах, интегрирующий множитель}
\subsubsection*{Уравнения в полных дифференциалах}
\begin{definition}
	Это уравнение
	\[M(x,\,y)dx + N(x,\,y)dy = 0\]
	называется уравнением в полных дифференциалах, если его левая часть является дифференциалом некоторой гладкой функции $F(x,\,y)$. Тогда это уравнение можно переписать в виде $dF(x,\,y) = 0$, так что его решение будет иметь вид
	\[F(x,\,y) = C\]
\end{definition}

\begin{proposition}
	Если функции $M(x,\,y)$ и $N(x,\,y)$ определены и непрерывны в некоторой односвязной области $\Omega$ и имеют в ней непрерывные частные производные по $x$ и по $y$, то изначальное уравнение будет уравнением в полных дифференциалах тогда и только тогда, когда выполняется тождество
	\[\frac{\partial M(x,\,y)}{\partial y} \equiv \frac{\partial N(x,\,y)}{\partial x}\]
\end{proposition}

\subsubsection*{Интегрирующий множитель}
Пусть дано уравнение в дифференциалах, которое не является уравнением в полных дифференциалах.

\begin{definition}
	Функция $\mu(x,\,y) \neq 0$ называется интегрирующим множителем для исходного уравнения, если уравнение
	\[\mu(x,\,y)(M(x,\,y)dx + N(x,\,y)dy) = 0\]
	является уравнением в полных дифференциалах. Отсюда следует, что функция $\mu$ удовлетворяет условию
	\[\frac{\partial(\mu M)}{\partial y} \equiv \frac{\partial(\mu N)}{\partial x}\]
	Это равенство даёт уравнение в частных производных первого порядка для $\mu(x,\,y):$
	\[N\frac{\partial\mu}{\partial x} - M\frac{\partial\mu}{\partial y} = \left(\frac{\partial M}{\partial y} - \frac{\partial N}{\partial x}\right)\mu\]
	Поделив обе части последнего уравнения на $\mu$, перепишем его в виде:
	\[N\frac{\partial \ln\mu}{\partial x} - M\frac{\partial \ln\mu}{\partial y} = \frac{\partial M}{\partial y} - \frac{\partial N}{\partial x}\]
\end{definition}

\subsubsection*{Точные и замкнутые 1-формы, лемма Пуанкаре}
\begin{definition}
	Форма $\omega$ называется точной, если существует гладкая функция $F$, такая что $\omega = dF$
\end{definition}

\begin{definition}
	Форма $\omega = F_1dx_1 + \ldots + F_mdx_m$ называется замкнутой, если
	\[\forall k,\,i:\: \frac{\partial F_i}{\partial x_k} = \frac{\partial F_k}{\partial x_i}\]
\end{definition}

\begin{definition}
	Область $\Omega \subseteq \mathbb{R}^m$ называется звёздной, если для некоторой точки $p \in \Omega$ и для любой другой точки $q \in \Omega$ отрезок $[p,\, q]$ полностью содержится в $\Omega$.
\end{definition}

\begin{lemma}
	В звёздной области любая замкнутая $C^1$-гладкая дифференциальная 1-форма точна.
\end{lemma}

\begin{proof}
	Будем считать, что точка $p$ из определения звёздной области находится в начале координат. Пусть $\omega$ -- замкнутая форма, $\omega = A_1dx_1 + \ldots A_ndx_n$.

	Заметим, что для любой точки $x = (x_1,\,\ldots,\,x_n)$ и любой функции $G:\: \mathbb{R}^n \to \mathbb{R}$
	\[G(x) - G(0) = \int_{[0,\,x]}dG = \int_0^1\left(x_1\frac{\partial G}{\partial x_1}(tx) + \ldots + x_n\frac{\partial G}{\partial x_n}(tx)\right)dt\]
	мы параметризовали отрезок $[0,\,x] \subset \mathbb{R}^n$ параметром $t$. Пользуясь этим равенством, можно восстановить любую функцию по набору её производных.

	Поэтому естественно определить $F$ таким образом:
	\[F := \int_0^1 \sum_{i = 1}^n x_iA_i(xt)dt\]
	Нам осталось проверить, что $\frac{\partial F}{\partial x_s} = A_s$. Действительно,
	\begin{align*}
		\frac{\partial F}{\partial x_s} = \int_0^1 \frac{\partial}{\partial x_s} \sum_{i = 1}^n x_iA_i(xt)dt = \int_0^1 A_s(tx) + \sum_{i = 1}^n x_i \frac{\partial}{\partial x_s}A_i(tx)dt = \\
		= \int_0^1 A_s(tx) + \sum_{i = 1}^n x_i \frac{\partial}{\partial x_i}A_s(tx)dt = \int_0^1 \frac{d}{dt}(tA_s(tx))dt = A_s(x)
	\end{align*}
	Итак, $dF = A_1dx_1 + \ldots + A_ndx_n = \omega$.
\end{proof}

\subsubsection*{Гамильтоновые векторные поля на плоскости}
\begin{definition}
	Пусть $H:\: \mathbb{R}^2 \to \mathbb{R} \in C^1(\mathbb{R}^2)$. Тогда векторное поле
	$\vec{v}: \begin{cases}
			\dot{x} = -\frac{\partial H}{\partial y}(x,\,y) \\
			\dot{y} = \frac{\partial H}{\partial x}(x,\,y)
		\end{cases}$ называется гамильтоновым тогда и только тогда, когда $\text{div }\vec{v} = 0$
\end{definition}

\subsection{Методы понижения порядка для некоторых простейших типов дифференциальных уравнений. Уравнения первого порядка, не разрешённые относительно производной.}
\subsubsection*{Методы понижения порядка дифференциальных уравнений.}
\begin{enumerate}
	\item Пусть $F(x,\, y^{(k)},\,\ldots,\,y^{(n + k)}) = 0$

	      Замена: $z = y^{(k)}$, сводим к уравнению $F(x,\, z,\, z',\,\ldots,\,z^{(n)}) = 0$
	\item Пусть $F$ явно не зависит от $x$: $F(y,\, y',\,\ldots,\,y^{(n)}) = 0$

	      Замена: $y$ -- новая независимая переменная, $y' = p = p(y)$, то есть $y''_{xx} = p'_x = p'y'=p'p$
	\item Обобщённо-однородное уравнение

	      Пусть $\exists m,\,k:\: \forall \lambda > 0:\: F(\lambda x,\, \lambda^m y,\, \lambda^{m - 1}y',\,\ldots,\,\lambda^{m - n}y^{(n)}) = \lambda^k F(x,\,y,\,y',\,\ldots,\,y^{(n)})$

	      Замена: $x = e^y,\, y = z(t)e^{mt}$
	\item Однородные уравнения

	      Пусть $\exists k:\: \forall \lambda > 0:\: F(x,\, \lambda y,\, \lambda y',\,\ldots,\, \lambda y^{(n)}) = \lambda^k F(x,\,y,\,y',\,\ldots,\,y^{(n)})$

	      Замена: $y' = z(x)y,\, y'' = (z(x)y)' = z'y + zy' = z'y + z^2y = y(z + z^2)$
\end{enumerate}

\subsubsection*{Уравнения первого порядка, не разрешённые относительно производной}
\begin{definition}
	Уравнение первого порядка, не разрешённое относительно производной -- это уравнение вида
	\[F(x,\,y,\,y') = 0\]
	где $F(x,\,y,\,y')$ -- заданная непрерывная функция в некоторой непустой окрестности $G \subseteq \mathbb{R}^3_{(x,\,y,\,p)}$ с декартовыми прямоугольными координатами $x,\,y,\,p$.
\end{definition}

\begin{note}
	В общем случае для решения уравнения применяется метод введения параметра, который позволяет свести решение исходного уравнения к решению некоторого уравнения первого порядка в симметричной форме.

	Сам метод: положим $y' = p$ и рассмотрим систему
	\begin{equation}
		\label{BASE}
		\begin{cases}
			F(x,\,y,\,p) = 0 \\
			dy = pdx
		\end{cases}
	\end{equation}
\end{note}

\begin{proposition}
	Проектирование $\pi$ поверхности $F(x,\,y,\,p) = 0$ на плоскость $(x,\,y)$ вдоль оси $p$ переводит траектории поля в интегральные кривые системы (\ref*{BASE}).

	В тех точках поверхности, где производная $\frac{\partial F}{\partial p} \neq 0$, отображение $\pi$ является локальным диффеоморфизмом.
\end{proposition}

\begin{definition}
	Точки поверхности $F(x,\,y,\,p) = 0$, в которых производная $\frac{\partial F}{\partial p} = 0$, называются особыми точками уравнения (\ref*{BASE}).
\end{definition}

\begin{definition}
	Множество всех особых точек называется криминантой, а её проекция на плоскость $(x,\,y)$ -- дискриминантной кривой уравнения (\ref*{BASE}).
\end{definition}

\begin{definition}
	Решение уравнения (\ref*{BASE}) называется особым, если его интегральная кривая является дискриминантной кривой.
\end{definition}

\begin{note}
	Решение уравнения (\ref*{BASE}) можно трактовать, как траектории движения по этой поверхности, задаваемого векторным полем
	\begin{equation}
		\label{PODN_POLE}
		\begin{cases}
			\dot{x} = \frac{\partial F}{\partial p}\\
			\dot{y} = p\frac{\partial F}{\partial p}\\
			\dot{p} = -\left(\frac{\partial F}{\partial x} + p\frac{\partial F}{\partial y} \right)
		\end{cases}
	\end{equation}
\end{note}

\begin{definition}
	Другой важной кривой является кривая перегибов, состоящая из всех точек поверхности $F(x,\,y,\,p) = 0$, в которых третья компонента поля (\ref*{PODN_POLE}) обращается в нуль. 
\end{definition}

\begin{proposition}
	Криминанта и кривая перегибов связаны некоторым двойственным соотношением: преобразование Лежандра $(x,\,y,\,p) \to (p,\, xp - y,\, x)$ переводит всякую интегральную кривую $\gamma$ уравнения (\ref*{BASE}) в интегральную кривую $\gamma^*$ сопряжённого уравнения $F(p,\, xp - y,\, x) = 0$
\end{proposition}

\end{document}